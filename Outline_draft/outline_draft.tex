%Outline draft Thesis

\documentclass[a4paper,12pt]{report}
\usepackage[utf8]{inputenc}
\usepackage{graphicx}
\usepackage{amsmath}
\usepackage{color}   %May be necessary if you want to color links
\usepackage{float}
\usepackage{caption}
\usepackage{wrapfig}
\usepackage{todonotes}
\usepackage{fancyhdr}
\usepackage[a4paper,width=150mm,top=25mm,bottom=25mm]{geometry}

\usepackage{hyperref}
\hypersetup{
    colorlinks=true, %set true if you want colored links
    linktoc=all,     %set to all if you want both sections and subsections linked
    linkcolor=blue,  %choose some color if you want links to stand out
}
\hypersetup{
    colorlinks,
    citecolor=black,
    filecolor=black,
    linkcolor=black,
    urlcolor=black
}

\graphicspath{{./Images/}}

\fancyhead{}
\fancyhead[LE,RO]{Chapter \thechapter}
\fancyfoot{}
\fancyfoot[LE,RO]{\thepage}

\renewcommand{\headrulewidth}{0.4pt}
\renewcommand{\footrulewidth}{0.4pt}

%opening
\title{Influence of timing cuts in Testbeam data and Simulation.}
\author{Eldwan Brianne}
\date{\today}

\begin{document}

\maketitle

\newpage

\tableofcontents

\chapter{Introduction}

\chapter{Particle Physics}
\section{The Standard Model}
\section{The International Linear Collider: a future $e^+e^-$ linear collider}
\subsection{Damping rings}
\subsection{Sources}
\subsection{Linac}
\subsection{The International Large Detector (ILD)}
\section{ILC Physics}
\subsection{Higgs Physics}
\subsection{Electroweak Physics}
\subsection{Top mass measurement}
\subsection{Beyond the Standard Model}
\section{Calorimetry and Particle Flow}
\subsection{Particle interaction with matter}
\subsection{Particle Flow Principle}
\section{Simulation of particle showers}
\subsection{Electromagnetic showers}
\subsection{Hadronic showers}
\subsection{Physics Lists}

\chapter{Fast simulation of ILD}
\subsection{SGV, a fast simulation of ILD}
\subsection{Particle Flow emulation}
\subsection{Comparison to Full simulation}

\newpage

\chapter{AHCAL Engineering prototype}
\section{Silicon Photomultiplier (SiPM)}
\section{The CALICE Analog Hadronic Calorimeter}
\subsection{Physics Prototype and Validation}
\subsection{The CALICE Engineering Analog Hadronic Calorimeter}
\section{Commissioning of the CALICE AHCAL}
\subsection{Commissioning Procedure}
\subsection{Noise Measurement}

\newpage

\chapter{Testbeam Analysis of the CALICE AHCAL}
\section{The CERN SPS beamline}
\section{Testbeam Setup}
\section{Simulation Setup}
\subsection{Digitization effects}
\subsection{Beam Profiles}
\section{Event Selection}
\subsection{MIP Selection}
\subsection{Electron Selection}
\subsection{Pion Selection}
\section{Energy Calibration of the AHCAL}
\subsection{Pedestal Calibration}
\subsection{MIP Calibration}
\section{Timing Calibration of the AHCAL}
\subsection{Calibration using MIPs}
\subsection{Electron Analysis and Validation}
\subsection{Pion Analysis}

\newpage

\chapter{ILD Simulation and Timing}
\section{Detector Simulation and framework}
\subsection{Detector Simulation}
\subsection{ILCSoft}
\section{Reconstruction chain}
\subsection{Tracking}
\subsection{Calorimeter Digitization}
\subsection{Pandora PFA}
\section{Influence of time cuts on cluster properties}
\subsection{Procedure}
\subsection{Impact on Cluster properties (width, energy, resolution and linearity)}

\newpage

\chapter{Conclusion and Outlook}

\newpage

\chapter{Appendix}

\newpage

\chapter{Bibliography}

\end{document}
