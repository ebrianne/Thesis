%!TEX root = main.tex
\chapter{Summary}

The International Linear Collider (ILC) is a future linear electron-positron collider with a center-of-mass energy of 250 GeV, upgradeable to 500 GeV. It will require an unprecedented jet energy resolution to reach the goals in the precision measurement of the Standard Model parameters. In order to achieve the jet energy resolution in the range of 3-4\%, the Particle Flow Algorithms (PFAs) concept combines the tracking and calorimeter measurements into the best jet energy measurement, by measurement each individual particle in a jet and using the best sub-detector resolution to measure the energy. PFAs need to correctly associate energy depositions in the calorimeters and tracks. This needs an unprecedented spatial resolution which can be achieved with high granular calorimeters.

The CALICE collaboration develops such high granularity calorimeter prototypes. One of the design, the analog hadron calorimeter (AHCAL), consists of plastic scintillator tiles of $3\times3$ cm$^2$ area, read out by Silicon Photomultipliers (SiPMs). Several prototypes have been built and tested in various beams as a proof-of-concept. Nowadays, the focus of such calorimeters is the scalability to a full linear collider detector by integrating the front-end electronics onto the active layers.

This thesis presents the first timing analysis of a large scale analog hadron calorimeter based on scintillator-SiPM technology. In the testbeam campaign at CERN in July 2015, the CALICE AHCAL technological calorimeter prototype was operated in muon, electron and pion beams in an energy range up to 90 GeV. The main challenges in this analysis are the enormous number of channels to be calibrated in energy and time, the understanding of the different features of the front-end electronics and the contamination of beam events with multi-particles of different types and energy.

As part of this thesis work, the commissioning procedure of the AHCAL boards (HBU), that were used in the testbeam at CERN, has been presented. Around 60 SPIROC chips have been tested manually with a yield of 84\%. It takes around 10 minutes per chip to be tested and it is expected to be reduced to under few minutes by automatizing the chip testing for the next prototype. In total, 3 EBU and 24 HBU boards have been commissioned. Around 7-8 hours are needed to perform the commissioning for old generation boards due to the high range of LED voltage needed to calibrate the SiPM gain. New generation boards and improvements in SiPM quality reduce the commissioning time under 1 hour per board.

Before comparing the recorded data to simulation, the energy scale calibration of the detector must be performed and validated. Thus, the energy scale calibration of the full testbeam prototype has been presented. A robust method for extracting the most probable value of a MIP has been developed to accommodate for the high number of channels (3744) to be calibrated in the AHCAL. Around 85\% of the channels could be calibrated with a calibration uncertainty between 1\% to 3\%. The energy scale calibration has been validated on single channel level with simulation. The agreement between data and simulation is satisfactory.

This thesis focuses on the timing development of hadronic shower in a highly granular calorimeter. This has been presented in this thesis with the timing analysis of testbeam data recorded in July 2015 at CERN. Firstly, the timing calibration of a full scale hadron calorimeter prototype has been shown. After calibration, a time resolution of around 5 ns for muons and 8 ns for electron and pion showers was achieved. The increase in the time resolution for electron showers is mainly due to a feature of the front-end electronic that induces a shift of the timing measurement as a function of the number of hits within a SPIROC2b ASIC. A detailed validation of the simulation is performed with muons and electromagnetic interactions, yielding an agreement between data and simulation within 10-20\%.

The analysis of the pion data recorded with the AHCAL technological prototype aims to improve our understanding of the time development of hadronic showers. First, the correlation between the hit energy and the hit time has been investigated. The data showed that late depositions are concentrated at low hit energies below 1.5 MIPs in iron absorber. Secondly, the hit time as a function of the hit distance to the shower axis has been looked at. It showed that mostly delayed timing hits are at a great distance from the shower axis. Finally, the calorimeter testbeam setup allows for the investigation of time correlations between layers. The results showed that time correlations are visible at short distance range ($\sim$18\%) as well as long distance range in smaller proportions ($\sim$3\%).

A detailed comparison has been performed with several physics lists in \geant. Overall, a good agreement is present between data and simulations within statistical and systematical uncertainties. The tracking of low energy neutrons in the HP package or other implementations like in QBBC show that they are needed to reproduce well the tail of the data which is otherwise generally over-estimated. Time correlations are reproduced in simulation but the proportion of hits in data and simulation differ quite significantly. This may be due to the selection of the data that does not reject efficiently multi-particle events. More data and investigations are needed to understand furthermore the time development of hadron shower in a full calorimeter.

At last, as part of this thesis, the application of timing cuts on the full ILD detector simulation was performed. This analysis aims to understand the effect of timing cuts on calorimeter energy response and energy resolution and the development of hadron shower through the calorimeter. The results showed that, assuming a perfect timing resolution, timing cuts affect the calorimeter linearity and resolution to a few percent level (up to 6\%). The radial development of a hadron shower showed that timing cuts are removing mostly outer hits from the shower. Timing cuts reduce the width of the shower without affecting much the energy resolution. The effect of timing cuts on the energy resolution is described by an increase of the constant term. This effect has been investigated and it is understood that the timing cut has a bias effect on hadronic shower by decreasing the hadronic response of a shower, such that the EM fraction becomes more important. Timing cuts act as a non-compensation effect in the response to hadronic showers thus degrading the energy resolution. Finally, different timing resolutions were assumed and showed that ideally timing cuts with timing resolution around 1 ns would help to greatly reduce the shower width without affecting much the energy resolution. Timing cuts could be used to improve separation of overlapping showers and as well it could improve the energy resolution with software compensation using time information. However, more studies should be done using timing information in order to provide an idea of the required time resolution.

Some of the difficulties encountered in this analysis comes from the fact that only a small part of the prototype was equipped with active layers and using various generations of SiPMs and readout electronics. However, a new prototype has been build in April 2018 and consists of 40 active layers using homogenous electronics. The new prototype is currently being tested at the CERN SPS facility in various beams. Such prototype provides very detailed images of events that can be used to provide a very good selection. Moreover, highly granular calorimeters are now interesting also in the context of LHC, where high granularity is mainly used to suppress pile-up events. In this case, a very good timing resolution, in the order of few picoseconds, could help to improve the pile-up rejection.
