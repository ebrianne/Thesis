%!TEX root = main.tex
\chapter{Introduction}

The \textit{International Linear Collider} (ILC) and the \textit{Compact Linear Collider} (CLIC) are concepts of future lepton colliders. As seen in the past, a lepton collider experiment would be complementary to the results from the \textit{Large Hadron Collider} (LHC) in terms of precision measurements and potential discoveries.

The LHC at CERN is the most energetic particle accelerator ever constructed. It is colliding protons at a center of mass energy up to $\sqrt{s}$ = 14 TeV. In 2012, the LHC discovered the \textit{Higgs boson} with the CMS and ATLAS experiments. This marked the accomplishment of a cornerstone of the \textit{Standard Model} of particle physics.

The ILC is planned to collide electrons and positrons at a center of mass energy up to $\sqrt{s}$ = 500 GeV. In order to achieve the best possible measurement precision at the ILC, unprecedented detectors resolutions are needed. This can be achieved by the concept of \textit{Particle Flow algorithms} (PFAs). The detectors are designed based on the PFA approach in order to achieve the best possible jet energy resolution of around 3-4\% \cite{ILC_TDR_Vol1}. The PFA concept aims to combine measurements of the tracking system and the calorimeters by measuring each individual particles in a jet using the best sub-detector measurement. This requires calorimeters systems with an unprecedented \textit{granularity}.

\textit{Precision timing measurement} have not been a key aspect in hadron calorimeters. However, in order to cope with high-rate collision experiments such as CLIC where bunches are separated by 0.5 ns \cite{CLIC_CDR}, timing measurements become important. It would permit the separation of out-of-time pile up events and it would reduce fake energy deposits.

In addition, precision timing measurements could act as a \textit{software compensation} met\-hod \cite{Benaglia2016}. The hadron calorimeter for the ILC detectors is non-compensating meaning that the hadronic res\-ponse of a shower is lower than the electromagnetic response. The fluctuations in electromagnetic and hadronic energy fraction degrade the energy resolution. The time development of a shower is correlated to the energy fluctuations in a hadron shower and therefore, would allow for an event-by-event correction in order to improve the energy resolution of the calorimeter.

This thesis discusses the ongoing development of a highly granular calorimeter within the CALICE collaboration. The work presented in this thesis is based on the analysis of testbeam data collected during the summer of 2015 at CERN with the CALICE analog hadron calorimeter (AHCAL) technological prototype which is a scintillator-based hadron calorimeter using Silicon photomultiplier (SiPM) readout.

A recapitulation of the Standard Model of particle physics is given and a short explanation of the Higgs mechanism is presented in Chapter \ref{chap:Theory}, giving the basis for the need of high precision measurement of the Standard Model parameters.

It is followed by a description of the International Linear Collider machine and the different detector concepts, the International Large Detector (ILD) and the Silicon Detector (SiD). Furthermore, some of the key aspects of the ILC physics program are presented in Chapter \ref{chap:FutureColliders}.

An introduction to the interactions of particles with matter as well as the principles of calorimetry are presented in Chapter \ref{chap:CaloPFA}. In addition, the Particle Flow concept and its requirements are introduced.

The first and second generation calorimeter prototypes within the CALICE collaboration, with a focus on scintillator-SiPM readout, are introduced and compared in the Chapter \ref{chap:CALICE_Det}. These calorimeter prototypes have for goal to demonstrate the proof-of-concept for Particle Flow and the scalability of the hardware to a full linear collider detector.

The particle shower models in \geant are presented in Chapter \ref{chap:G4Simulation} along with the AHCAL simulation geometry model implementation and digitization procedure. This is particularly important to understand the fundamental basis of the simulation shower models as it will be compared to testbeam data.

The commissioning procedure of the AHCAL technological prototype is introduced and discussed in Chapter \ref{chap:Commissioning}.

The testbeam analysis results obtained as part of this thesis are presented in Chapters \ref{chap:EvtSelection} to \ref{chap:TimingPions}. The thesis uses testbeam data from the AHCAL technological prototype in the energy range of 10 to 90 GeV, recorded at the Super Proton Synchrotron (SPS) facility at CERN in July 2015. The goals of this analysis are to demonstrate the feasibility of the timing calibration of a full scale hadron calorimeter and its current timing performance in various beams and to improve our understanding of the time development of hadron showers. Firstly, the testbeam setup for this thesis and event selections are described (Chapter \ref{chap:EvtSelection}). Secondly, the energy scale calibration of the AHCAL is performed and the results are discussed (Chapter \ref{chap:ECalibAHCAL}). Thirdly, the timing analysis of the testbeam data is presented. This analysis includes the timing calibration of the full detector and the validation of the calibration with electron data (Chapter \ref{chap:TimingCalib} and \ref{chap:TimingValidation}). Finally, the analysis of pion showers data is discussed (Chapter \ref{chap:TimingPions}). The results of this analysis are compared to several \geant v10.1 physics lists.

Finally, the effects of timing cuts in the hadronic calorimeter of the full ILD detector simulation are investigated and discussed in Chapter \ref{chap:ILDTiming}. This analysis has for goal to understand the effects of timing cuts on the calorimeter energy response and energy resolution as well as their impact on the spatial development of a hadron shower.
