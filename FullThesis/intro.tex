%!TEX root = main.tex
\chapter{Introduction}

The \textit{Large Hadron Collider} (LHC) at CERN is the highest energy particle accelerator ever constructed. It is colliding protons at a center of mass energy up to $\sqrt{s}$ = 14 TeV. In 2012, the LHC discovered the \textit{Higgs boson} with the CMS and ATLAS experiments. This marked the accomplishment of a cornerstone of the \textit{Standard Model} of particle physics.

As seen in the past, a lepton collider experiment would be complementary to the results from the LHC in terms of precision measurements and potential discovery. Several concepts of future linear collider experiments exist, the \textit{International Linear Collider} (ILC) and the \textit{Compact Linear Collider} (CLIC). The ILC is planned to collide electrons and positrons at a center of mass energy up to $\sqrt{s}$ = 500 GeV. In order to achieve the best possible measurement precision at the ILC, unprecedented detector resolutions are needed. This can be achieved by the concept of \textit{Particle Flow algorithms}. The detectors are designed based on this approach in order to achieve a jet energy resolution of around 3-4\%. The PFA concept aims to combine measurements of the tracking system and the calorimeters by measuring each individual particles in a jet using the best sub-detector measurement for each particle. This requires calorimeters systems with an unprecedented \textit{granularity}.

Such calorimeter designs also have been selected for the high luminosity upgrade of the LHC. To maintain and improve the current performance of the LHC detectors under the challenging conditions of the HL-LHC, an increase of the longitudinal and lateral segmentation of the calorimeters is needed. While \textit{precision timing measurement} have not been a key aspect in hadron calorimeters, such measurements would allow to identify pile-up jets and would improve the vertex location of di-photon events. A measurement in the order of 20-30 ps (equivalent to around 1 cm separation) would allow the association of energy depositions in the calorimeter with vertices and would significantly reduce fake objects and energy deposits \cite{CMSCollaboration:2015zni}. This is even more important for high rate collision experiments such as CLIC where bunches are separated by 0.5 ns. Moreover, precision timing measurements could act as \textit{software compensation} method. It would improve the calorimeter energy resolution by identifying the different component of a hadronic shower and weighting the calorimeter response accordingly \cite{Benaglia2016}.

This thesis discusses the ongoing development of such highly granular calorimeters within the CALICE collaboration. It focuses on the technological prototype of a scintillator-based hadron calorimeter using Silicon photomultiplier (SiPM) readout designed for a future linear collider experiment. This thesis work is centered on the analysis of testbeam data collected during the summer of 2015 at CERN with the CALICE AHCAL technological prototype.

In Chapter 1, the theoretical foundations for this thesis are summarized. In Chapter 2, the International Linear Collider, the detector concepts and key aspects of the ILC physics program are presented. In Chapter 3, the principles of calorimetry are presented. Furthermore, the concept of Particle Flow Algorithms is explained. The first and second generation calorimeter prototypes within the CALICE collaboration, with a focus on scintillator-SiPM readout, are introduced and compared in Chapter 4. In Chapter 5, the particle shower models in \geant are presented along with the AHCAL simulation model implementation and digitization procedure. The commissioning procedure of the AHCAL technological prototype is introduced and discussed in Chapter 6. In Chapter 7, the testbeam setup for this thesis and event selections are described. The energy scale calibration of the AHCAL is discussed in Chapter 8. In Chapter 9 and 10, the timing analysis of the testbeam data is presented. This analysis includes the timing calibration of the full detector and the validation of the calibration with electron data. In the end, the results of the analysis of pion showers is discussed and compared to several \geant v10.1 physics lists. Finally, in Chapter 11, the effects of timing cuts in the hadronic calorimeter of the full ILD detector simulation are investigated and discussed.
