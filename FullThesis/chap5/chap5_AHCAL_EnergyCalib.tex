\chapter{Energy Calibration of the AHCAL}
\label{chap:ECalibAHCAL}

The CALICE Analog Hadronic Calorimeter (AHCAL) technological prototype was installed at the SPS CERN facilities in July and August 2015, in order to provide energy and time measurements of electromagnetic and hadronic showers using plastic scintillators. The data recorded in each cells of the calorimeter is measured in ADC counts, thus this scale cannot be compared directly between different channels. Therefore to compare them, all channels have to be scaled to a common physical energy unit. For the AHCAL, the Minimum Ionizing Particle or MIP unit is chosen. This unit relates to the cell energy in a well and understood physical process almost independent to outside conditions.

The conversion requires a calibration of each cells of the calorimeter which is by itself a challenge due to the high number of readout channels. In this testbeam, 3744 channels have to be calibrated. Due to the boards equipped with SiPMs from various different batches, the procedure needs to be automatic and robust to extract the calibration constant for each channel. At the end, all the calibration constants are entered in the official CALICE database.

This chapter will firstly describe the beamline facilities used in July 2015 at CERN, followed by the description of the tesbeam setup and finally describe the procedure performed for the AHCAL energy calibration.

\section{Beamline Setup}

\section{TestBeam Setup}

\section{Energy Calibration of the AHCAL}
