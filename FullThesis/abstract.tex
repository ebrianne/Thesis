%!TEX root = main.tex
\cleardoublepage
\phantomsection
\addcontentsline{toc}{chapter}{Abstract}

\thispagestyle{empty}
\begin{center}
{\bf Abstract}
\end{center}

This thesis presents the study of the time development of hadronic showers in a highly granular hadron calorimeter using data collected by the CALICE Analog Hadron Calorimeter (AHCAL) technological prototype at the Super Proton Synchrotron (SPS) at CERN in July 2015 in $\mu^-$, $e^-$ and hadron beams up to 90 GeV.

Highly granular calorimeters are designed based on the concept of Particle Flow Algorithm which aims to measure each individual particles in a jet using the best sub-detector measurement. The Particle Flow approach is a key aspect of the detectors for the International Linear Collider (ILC) which is a future linear lepton collider designed for precision physics studies at a center of mass energy up to 500 GeV.

The CALICE Collaboration is developing highly granular calorimeters for the ILC. The AHCAL is one of the detector concepts based on $30\times30\times3$ mm$^3$ scintillating tiles read out by Silicon Photomultipliers (SiPM). The SiPM signal is processed by an ASIC (SPIROC2b) capable to measure the amplitude and the time of individual calorimeter hits. A second generation technological prototype of the AHCAL is developed to focus on the full scalability of the detector. The prototype used in this thesis is composed of 14 active layers corresponding to 3744 channels inserted into an steel absorber structure of 4 $\lambda_i$ depth. For the first time, a large scale prototype has been operated successfully in various beams.

This analysis concentrates on the timing capabilities of the AHCAL prototype. In a first part, the challenging timing calibration procedure, due to the large amount of channels and features of the front-end, is presented. A timing resolution of 5 ns for muons and 8 ns for electron and pion showers has been achieved. In a second part, this thesis shows the study of the time development of hadron showers with the AHCAL prototype by investigating timing correlations with the hit energy, the hit distance to the shower axis and between layers. The results of the testbeam data have been compared to various simulations.

In the last part of this thesis, a study of the effects of timing cuts on calorimeter response, energy resolution and shower topology is presented using a full GEANT4-based simulation of the International Large Detector (ILD) concept. Using the timing information of individual calorimeter hits has the potential to improve pattern recognition and calorimeter energy resolution.

\newpage
\thispagestyle{empty}
\begin{center}
{\bf Zusammenfassung}
\end{center}
\lipsum[23]
