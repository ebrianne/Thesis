%!TEX root = main.tex
\cleardoublepage
\phantomsection
\addcontentsline{toc}{chapter}{Abstract}

\thispagestyle{empty}
\begin{center}
{\bf Abstract}
\end{center}

This thesis presents the study of the time development of hadronic showers in a highly granular hadron calorimeter using data collected by the CALICE Analog Hadron Calorimeter (AHCAL) technological prototype at the Super Proton Synchrotron (SPS) at CERN in July 2015 in $\mu^-$, $e^-$ and hadron beams up to 90 GeV.

Highly granular calorimeters are designed for the concept of Particle Flow Algorithm which aims to measure each individual particles in a jet using the best sub-detector measurement. The Particle Flow approach is a key aspect of the detectors for the International Linear Collider (ILC) which is a future linear lepton collider designed for precision physics studies at a center of mass energy up to 500 GeV.

The CALICE Collaboration is developing highly granular calorimeters for future colliders. The AHCAL is one of the detector concepts based on $30\times30\times3$ mm$^3$ scintillating tiles read out by Silicon Photomultipliers (SiPM). The SiPM signal is processed by an ASIC (SPIROC2b) capable to measure the amplitude and the time of individual calorimeter hits. A second generation technological prototype of the AHCAL is developed to focus on the full scalability of the detector. The prototype used in this thesis is composed of 14 active layers corresponding to 3744 channels inserted into an steel absorber structure of 4 $\lambda_i$ depth. For the first time, a large scale prototype has been operated successfully in various beams.

This analysis concentrates on the timing capabilities of the AHCAL prototype. In a first part, the challenging timing calibration procedure, due to the large amount of channels and features of the front-end, is presented. A timing resolution of 5 ns for muons and 8 ns for electron and pion showers has been achieved. In a second part, this thesis shows the study of the time development of hadron showers with the AHCAL prototype by investigating timing correlations with the hit energy, the hit distance to the shower axis and between layers. The results of the testbeam data have been compared to various simulations.

In the last part of this thesis, a study of the effects of timing cuts on calorimeter response, energy resolution and shower topology is presented using a full GEANT4-based simulation of the International Large Detector (ILD) concept. Using the timing information of individual calorimeter hits has the potential to improve pattern recognition and calorimeter energy resolution.

\newpage
\thispagestyle{empty}
\begin{center}
{\bf Zusammenfassung}
\end{center}

Diese Arbeit präsentiert die Untersuchung der zeitlichen Entwicklung hadronischer Schauer in einem hochgranularen Hadronenkalorimeter unter Verwendung von Daten, die mit dem CALICE Analog Hadron Calorimeter (AHCAL) Technologieprototyp am Super Proton Synchrotron (SPS) am CERN im Juli 2015 in $\mu^- $, $e^-$ und Hadronenstrahlen bis zu 90 GeV.

Hoch granulare Kalorimeter wurden für das Konzept des Partikelflussalgorithmus entwickelt, mit dem jede einzelne Partikel in einem Jet mit der besten Sub-Detektor-Messung gemessen werden soll. Der Particle-Flow-Ansatz ist ein Schlüsselaspekt der Detektoren für den International Linear Collider (ILC), ein zukünftiger linearer Lepton-Collider, der für Untersuchungen der Präzisionsphysik in einem Schwerpunkt der Massenenergie von bis zu 500 GeV entwickelt wurde.

Die CALICE Collaboration entwickelt hochgranulare Kalorimeter für zukünftige Collider. Der AHCAL ist eines der Detektorkonzepte, basierend auf $30\times30\times3$ mm$^3$ szintillierenden Kacheln, die von Silicon Photomultipliers (SiPM) ausgelesen werden. Das SiPM-Signal wird von einem ASIC (SPIROC2b) verarbeitet, der die Amplitude und die Zeit einzelner Kalorimetertreffer messen kann. Ein technologischer Prototyp der zweiten Generation des AHCAL wurde entwickelt, um sich auf die volle Skalierbarkeit des Detektors zu konzentrieren. Der in dieser Arbeit verwendete Prototyp besteht aus 14 aktiven Schichten, die 3744 Kanälen entsprechen, die in eine Stahlabsorberstruktur von 4 $\lambda_i$ Tiefe eingesetzt sind. Zum ersten Mal wurde ein großformatiger Prototyp erfolgreich in verschiedenen Strahlen betrieben.

Diese Analyse konzentriert sich auf die Timing-Fähigkeiten des AHCAL-Prototyps. In einem ersten Teil wird das anspruchsvolle Zeitkalibrierungsverfahren aufgrund der großen Anzahl von Kanälen und Merkmalen des Front-Ends vorgestellt. Eine zeitliche Auflösung von 5 ns für Myonen und 8 ns für Elektronen- und Pionschauer wurde erreicht. In einem zweiten Teil dieser Dissertation wird die Zeitentwicklung von Hadronschauern mit dem AHCAL Prototyp untersucht, indem zeitliche Korrelationen mit der Trefferenergie, der Trefferentfernung zur Duschachse und zwischen Schichten untersucht werden. Die Ergebnisse der Teststrahldaten wurden mit verschiedenen Simulationen verglichen.

Im letzten Teil dieser Arbeit wird eine Studie über die Auswirkungen von Timing-Kürzungen auf die Kalorimeterantwort, die Energieauflösung und die Duschtopologie mit einer vollständigen GEANT4-basierten Simulation des International Large Detector (ILD) -Konzepts vorgestellt. Die Verwendung der Zeitinformationen der einzelnen Kalorimetertreffer hat das Potenzial, die Mustererkennung und die Energieauflösung der Kalorimeter zu verbessern.
