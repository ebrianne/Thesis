%!TEX root = ../main.tex
\chapter{CALICE Calorimeter concepts}

The CALICE Collaboration is developing and testing electromagnetic and hadronic calorimeters. These detector concepts are optimized toward a linear collider environment such as the International Linear Collider (ILC) \cite{Behnke:2013lya} or Compact Linear Collider (CLIC) \cite{2012arXiv1202.5940L} but collaboration with the Large Hadron Collider community for the High-Lumi upgrade (HL-LHC) is ongoing \cite{1748-0221-12-01-C01042}. The CALICE calorimeters are high granularity calorimeter optimized for the particle flow algorithm (Pandora PFA) providing a very detailed image of physics events.\\
Several prototypes for physics were build in the past and tested in testbeam campaigns at DESY, CERN and FNAL \cite{1748-0221-3-08-P08001, 1748-0221-5-05-P05004, 1707.07126v2, 1748-0221-10-10-P10039, 1748-0221-3-05-P05001}. Three physics prototypes of 1 m$^3$ were conceived using different active material and absorbers as well different readout schemes.\\
Nowadays, the CALICE Collaboration is still performing analysis on the data collected by these prototypes \cite{OskarCAN, YasmineCAN}. Also, the collaboration is looking forward in the integration into a full linear collider detector by designing several new technological calorimeter prototypes.\\
In this chapter, the electromagnetic and hadronic calorimeter concepts will be introduced, described and compared in details.

\section{Electromagnetic Calorimeters}

The CALICE Collaboration is developing two different electromagnetic calorimeters (ECAL) concepts. Physics prototypes whose goal was to prove the performance of such calorimeters for detailed measurements of EM showers, were first built. Now engineering prototypes are designed in order to improve the calorimeter design, the integration of the front-end electronics and the readout scheme. In the next subsections, the silicon-based SiECAL and the scintillator-based ScECAL calorimeters using both tungsten as absorber material will be described.

\subsection{SiECAL}

The SiECAL physics prototype consists of 30 active and absorber layers. The sensitive layer is made of high-resistivity silicon wafers 525 \si{\micro\meter} thick. These are divided in 6$\times$6 cm$^2$ sensors, segmented into a matrix of 1$\times$1 cm$^2$ PIN diodes operated in reversed bias. The total active area is 18$\times$18 cm$^2$ which resulted in 9720 channels. The depth of the calorimeter was 24 X$_0$ achieved by 10 layers of 0.4 X$_0$ (1.4 mm), followed by 10 layers of 0.8 X$_0$ (2.8 mm) and 10 more layers of 1.2 X$_0$ (4.2 mm) thick tungsten absorber plates. No front-end electronics was integrated into the layers but placed off-detector using analog lines. The performance of such calorimeter was tested in various beams at DESY and CERN. An energy resolution of $\frac{16.53\%}{\sqrt{E}}$ stochastic term and $1.07\%$ constant term was achieved \cite{ADLOFF2009372}.

\begin{figure}[htbp!]
  \centering
  \begin{subfigure}[t]{0.49\textwidth}
    \includegraphics[width=1.\linewidth]{chap3/fig/3DProtoH.png}
    \caption{} \label{fig:SiWECALPhysics}
  \end{subfigure}
  \hfill
  \begin{subfigure}[t]{0.49\textwidth}
    \includegraphics[width=1.\linewidth]{chap3/fig/SiW-ECAL_Techno.png}
    \caption{} \label{fig:SiWECALTechno}
  \end{subfigure}
  \caption{\subref{fig:SiWECALPhysics}) The schematics of the SiW-ECAL physics prototype. \subref{fig:SiWECALTechno}) Picture of a layer of the SiW-ECAL technological prototype.}
\end{figure}

After the validation of the calorimeter concept, a technological SiECAL prototype is being developed focusing on the integration into a full linear collider detector. To do this, modules close to the ILD design are being developed taking into account mass-production and low-power front-end electronics are integrated into the detector volume. The silicon wafers are larger and divided into 9$\times$9 cm$^2$ sensors. The PIN diode matrix is reduced to 5$\times$5 mm$^2$ pads to improve the pattern recognition of the calorimeter. New designs to the sensor are also made to minimize dead area at the sensor edge and cross-talk effects. The front-end is equipped with an ASIC, the SKIROC2 chip \cite{1748-0221-6-12-C12040}. It has 64 channels with adjustable gain charge pre-amplifier, a 12-bit ADC and digital logic. Also, it allows for auto-triggering with adjustable threshold (below 0.5 MIP) and hit time recording performed on a 12-bit TDC ramp. The SKIROC2 ASIC is designed to match the ILC beam structure (see section \ref{}) and thus allows for a power-pulsed mode where electronics are switched off between ILC bunches. This allows a very low power dissipation in the order of 25 \si{\micro\watt} per channel.

\subsection{ScECAL}
\label{subsec:ScECAL}

The ScECAL physics prototype consists of 30 layers of scintillator and tungsten carbide absorber plates 3.5 mm thick. The total calorimeter thickness is 266 mm or 21.5 X$_0$. The layers have a transverse area of 180$\times$180 mm$^2$. Each layer is composed of four rows of 18 scintillator strips of dimensions 45$\times$10$\times$3 mm$^3$ and the strips are placed orthogonally in consecutive layers. The strips have a Wavelength-shifting Fiber (WLS) inside to guide the scintillation light to a Silicon-Photomultiplier (SiPM), a MPPC from Hamamatsu. This accounts for 2160 channels to be read out. This prototype was tested in various beams, an energy resolution of 12.6\% stochastic and 1.6\% constant term was demonstrated \cite{1707.07126v2}.

To look forward, a technological prototype is now developed to accommodate the front-end electronics into the layer to reduce the amount of dead material due to cabling. An ECAL Base Unit (EBU) has 144 scintillator strips of dimensions 45$\times$5$\times$2 mm$^3$. Each layer has a transverse dimension of 180$\times$180 mm$^2$. The strips don't have a WLS fiber due to improvements in SiPM technology for blue light ($\sim$ 450 nm) detection efficiency. Several designs in SiPM and scintillator strip shape are being studied to optimize light collection. In this thesis (see section \ref{}), two designs were used in testbeam at CERN, a bottom-side readout and baseline readout design. The former uses 10k pixels surface-mounted MPPC, the latter uses 1.6k pixels MPPC placed on the side of the strips. Each SiPM are read out by an ASIC, the SPIROC2B. Each layer are equipped with four SPIROC ASICs. Each channel has also an integrated LED calibration system in order to monitor the SiPM gain.

\subsubsection{SPIROC2B ASIC}

The SPIROC2B (SiPM Integrated Read-Out Chip) is a dedicated ASIC to readout and digitize the signal of 36 SiPM channels. Each channel can be tuned in bias voltage from -4.5V to 0V with a resolution of around 20-200 mV. Each channel, also, has a configurable charge pre-amplifier gain to cover a high SiPM pixel dynamic range. Each channel is equipped with a capacitor-array, called memory-cells, with a depth of 16 events. A 12-bit Wilkinson ADC is used to digitize the charge stored in the capacitor array for the amplitude and time measurement. The hit time is registered via a 12-bit TDC voltage ramp with a designed resolution of 100 ps if operated in ILC-like conditions (ramp length of 200 \si{\nano\second}), in testbeam, the theoretical time resolution is around 1.9 ns (ramp length of 4 \si{\micro\second}). The chip can be operated in power-pulsing mode where parts of the chips not needed in any given state of operation can be switched off.

\begin{figure}[htbp!]
  \centering
  \includegraphics[width=0.7\linewidth]{chap3/fig/SPIROC2B_schematic.png}
  \caption{Schematic of the analog signal path of the SPIROC2 for a single channel \cite{SPIROC2_datasheet}.} \label{fig:SPIROC2B_sche}
\end{figure}

The chip can be operated in either external trigger mode (ET) or auto-trigger mode (AT). In external trigger, the signal of each cell is sampled synchronously to an external signal. In auto-trigger mode, the signal of each cell is compared to a configurable threshold, if the signal is above the threshold, the signal is sampled. Once the 16 memory-cells are filled, no further hits can be stored. Thus, the memory-cells are readout, digitized and the data is transferred out of the chip.

\section{Hadronic Calorimeters}

\subsection{AHCAL}
\label{subsec:SPIROC2B}

\subsection{SDHCAL}
\subsection{DHCAL}
