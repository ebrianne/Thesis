%!TEX root = ../main.tex
\chapter{Timing studies in ILD}

The chapter \ref{chap:TimingPions} showed that hadronic showers have several components in their time development. Early hit times come most likely from the electromagnetic core of the shower and ionizing particles and the late hit times from neutrons and late processes in the de-excitation of nuclear states. Hit times could then be used in order to separate the electromagnetic part of the shower from the hadronic part and therefore could be a tool to improve the pattern recognition in particle flow. This chapter will look at the application of timing cuts for the ILD detector and its impact on hadronic showers.\\

In this chapter, a study of timing cuts on hadronic shower is performed. The goal of this study is to assess the influence of timing cuts on the properties of hadronic showers as for example the width of the shower as well as the needed time resolution. The study will be divided into 2 parts, the first part assuming a perfect time resolution and the second part assuming time resolution for different cases.

\section{Simulation and software framework}

\subsection{ILD Detector Simulation}

The following analysis is using one of the generic ILD detector models (ILD\_o1\_v05) as described in chapter \ref{sec:ILD} within the \mokka framework. Many other models are also considered for ILD as shown in table \ref{table:ILDOptions}. \mokka is a front-end to \geant and provides a realistic geometry of the ILD detector. The \mokka version used is v08-05 and the \geant version is 10.01.
The simulation is performed by using the particle gun provided in \geant to shoot particles (\piminus or \kzeroL) with a fixed energy in different regions of the detector by randomly varying the angles $\theta$ and $\phi$ of the gun. To model hadronic showers, the QGSP\_BERT\_HP physics list was used. The output of the simulation provides a \lcio file containing collections of the tracking hits and simulated calorimeter hits. This file is then reconstructed within the \marlin framework.

\begin{table}[htb!]
  \centering
  \caption{Considered ILD detector options.} \label{table:ILDOptions}
  \begin{tabular}{|c|c|c|}
    \hline
    Option & ECAL Technology & HCAL Technology \\
    \hline
    ILD\_o1\_v05 & SiW-ECAL & AHCAL \\
    ILD\_o2\_v05 & SiW-ECAL & SDHCAL \\
    ILD\_o3\_v05 & Sc-ECAL & AHCAL \\
    \hline
  \end{tabular}
\end{table}

\section{Reconstruction chain}
\label{sec:recochain}

The reconstruction is done on simulated data in order to implement detector effects. For example, calorimeter hits need to be digitized by implementing threshold and readout effects.

\subsection{Tracking}

The tracking reconstruction is performed on each individual tracking detector. Track segments are identified by pattern recognition algorithms.

Track fitting is performed using the track segments with an Kalman filter \cite{Fruhwirth:1987fm} to identify trajectories of charged particles. Each track contains origin, direction, charge and momentum of the particle.

\subsection{Calorimeter digitization}
\label{subsec:ILDDigiCalo}

The calorimeter digitization is performed on simulated calorimeter hits as part of the ILDCaloDigi processor \cite{Jeans2015}. It accounts for threshold effects from the electronics, the sampling fraction of the calorimeter and the readout technology used. In the considered model of ILD, the SiW-ECAL and AHCAL are used.

For the AHCAL, it uses a SiPM based technology. The digitization then takes into account the finite number of pixels that can be fired as well as the statistical fluctuations related to pixel readout \cite{Hartbrich:2016bbz}.

Concerning time, it uses a simple digitization. For a simulated hit, all sub-hit contributions are looped over. Only sub-hits such as $t_{subhit} - t_{ToF} < t_{cut}$ ($t_{cut}$ default value is \SI{100}{\ns}) are added to create a digitized calorimeter hit. The time of the digitized calorimeter hit is the time of the first sub-hit contributing to it. This is done to simulate the time acceptance of the readout electronics. This modelization of timing is very simplified as in reality the electronics are shaping the signal with a certain shaping time and register the time of the first contribution over the threshold (default is 0.5 MIP).

\subsection{Pandora PFA}

PandoraPFA \cite{Thomson:2009rp} is the Particle Flow algorithm used for Linear Colliders as explained in chapter \ref{sec:PFA}. It uses as input tracks and calorimeter hits to form Particle Flow Objects (PFO). It uses a complex multi-stage process but basically, calorimeter hits are clustered and associated to tracks (if any). Then a re-clustering step is merging or splitting clusters in order to match the track energy if any. If the right criteria are matched, it forms a PFO which contains information about the reconstructed objects.
