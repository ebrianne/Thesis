\chapter{Timing study on ILD detector simulation}

Simulation of detector response is an essential part in high energy physics experiment. In Early stage of a project, simulations are done in order to explore and understand the possibilities of a detector design as well as its limitations. Simulation can be use as a way to determine requirements of an experiment to reach certain goals. During data-taking and afterwards, simulations are used model physics processes to compare  the expected value from theory to a measured value for various processes.
In this chapter, software tools will be briefly introduced. The \ilcsoft framework used for this analysis will be described in \ref{}. The chain starts by the simulation of single kaons $(K^{0}_{L})$ interaction with the ILD detector model based on \geant. Then simulated events undergoes the full chain reconstruction as explained in \ref{}. The procedure of the analysis (based on Marlin) and its conclusions will be presented in \ref{}.

\section{Simulation and software framework}

\subsection{\ilcsoft software framework}

Various tools developed by the Linear Collider community is regrouped in a common software framework called \ilcsoft \cite{ILCSOFT}. It provides a complete framework that can be used for Monte-Carlo studies and experiments. As an example, physics studies, ILD detector optimisation and performance for the ILC are performed under the \ilcsoft framework.

Most of the tools in the framework use an Event Data Model (EMD) named Linear Collider I/O (\lcio) which provides a reliable and performant solution for simulation and analysis studies \cite{LCIO}. With this tool, various detector concepts and analysis can be shared.

The \ilcsoft framework provides a modular \cpp framework named \marlin for reconstruction and analysis of physics events \cite{Marlin}. \marlin uses \lcio seamlessly and is configured using XML steering files. \marlin enables users to develop custom modules for their own and run it along other already existing modules.

The reconstruction and analysis tools used in this analysis are mostly part of \ilcsoft. For this thesis, \ilcsoft v01-17-11 was used for simulation, reconstruction and analysis.

\subsection{ILD Detector Simulation}

The following analysis is using one of the generic ILD detector model (ILD\_o1\_v05) as describe in \ref{} within the \mokka framework. Many other models are also considered for ILD as shown in Table \ref{ILDOptions}. \mokka is a front-end to \geant and provides a realistic geometry of the ILD detector. The \mokka version used is v08-05 and the \geant version is 10.01.
The simulation is performed by simply using the particle gun provided in \geant to shoot particles ($\pi^{-}$ or $K^{0}_{L}$) in different regions of the detector by randomly variating the angles $\theta$ and $\phi$ of the gun. To model hadronic showers, the QGSP\_BERT physics list was used. The output of the simulation provides a lcio file containing collections of the tracking hits and simulated calorimeter hits. This file is then reconstructed within \marlin.

\begin{center}
  \begin{tabular}{|c|c|c|}
    \hline
    Option & ECAL Technology & HCAL Technology \\
    \hline
    ILD\_o1\_v05 & SiW-ECAL & AHCAL \\
    ILD\_o2\_v05 & SiW-ECAL & SDHCAL \\
    ILD\_o3\_v05 & Sc-ECAL & AHCAL \\
    \hline
  \end{tabular}
  \medskip
  \captionof{table}{Considered ILD detector options.}
  \label{ILDOptions}
\end{center}

\section{Reconstruction chain}

The reconstruction is done on simulated data in order to implement detector effects. For example, calorimeter hits need to be digitized by implementing threshold and readout effects.

\subsection{Tracking}

The tracking reconstruction is perform on the VTX and TPC hits. First the simulated hits are smeared by the foreseen resolution of \SI{3}{\micro\meter} for the VTX and \SI{50}{\micro\meter} in R$\phi$ and \SI{400}{\micro\meter} in z for the TPC as well takes into account detector geometry and space charge effects (diffusion, EM-distorsions).

In a second step, track fitting is performed on hits based on an inversed Kalman-filter. The seed of the track starts from the outer part of the tracker and adds hits by going back inside the tracker until it reaches the vertex.

\subsection{Calorimeter Digitization}

The calorimeter digitisation is performed on simulated calorimeter hits \cite{ILDCaloDigiPaper}. It takes account for threshold effect from the electronics, sampling fraction of the calorimeter and the readout technology used. In the considered model of ILD, the SiW-ECAL and AHCAL are used.

In both cases, it uses a silicon-pixel based technology. The digitisation then takes into account the finite number of pixels that can be fired as well as the statistical fluctuations related to pixel readout \cite{OskarThesis}.

\subsection{Pandora PFA}

\section{Influence of time cuts on shower properties}

\subsection{Procedure}
\subsection{Impact on hadronic shower properties}
