\chapter{Timing study on ILD detector simulation}

Simulation of detector response is an essential part in high energy physics experiment. In Early stage of a project, simulations are done in order to explore and understand the possibilities of a detector design as well as its limitations. Simulation can be use as a way to determine requirements of an experiment to reach certain goals. During data-taking and afterwards, simulations are used model physics processes to compare  the expected value from theory to a measured value for various processes.\\

In this chapter, software tools will be briefly introduced.

\section{Detector Simulation and framework}

\subsection{Detector Simulation}
\subsection{ILCSoft framework}

\section{Reconstruction chain}

\subsection{Tracking}
\subsection{Calorimeter Digitization}
\subsection{Pandora PFA}

\section{Influence of time cuts on shower properties}

\subsection{Procedure}
\subsection{Impact on hadronic shower properties}
