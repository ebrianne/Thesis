%!TEX root = ../main.tex
\section{Conclusion and Outlook}

In this chapter, a first look at the effect of timing cuts on \geant simulation for the ILD has been done. In a first step, a perfect time resolution was assumed and shows that timing cuts affect the calorimeter linearity and resolution to a few percent level (up to 6\% for 5 GeV kaon). It has been shown that timing cuts are removing mostly outer hits from the hadronic shower thus reducing the width of the shower without affecting much the energy resolution. This could be used as a starting point in order to improve clustering and pattern recognition in PandoraPFA. The effect of timing cuts on the energy resolution is mostly an increase of the constant term. This has been investigated and it is understood that the timing cut has a bias effect on hadronic shower by cutting away late hits mostly coming from the hadronic part of the shower and as the EM and hadronic fraction of a shower fluctuates a lot on an event-by-event basis. Timing cuts act as a non-compensation effect in the response to hadronic showers thus degrading the energy resolution.

Finally, a time smearing between 0.4 to 8 ns was applied to the simulation corresponding to the foreseen, ideal and currently achieved time resolution for the AHCAL. It shows that a time resolution in the order of 1 ns and below does not affect much the linearity and resolution of the calorimeter. The energy resolution degrades between 6-8\% for a tight timing cut and reduces the shower width by 40\%. This time resolution would be ideal to be able to use the information at a later stage to improve the calorimeter resolution. As for the currently achieved time resolution of 8 ns, it is not ideal but could still to a certain level provide information that could be used to improve shower separation.

In the future, analysis using time information should be performed in order to evaluate the potential of timing and the needed time resolution in order to help in the separation of overlapping nearby hadronic showers and improve energy resolution by software compensation.
