%!TEX root = ../main.tex
\section{Conclusion and Outlook}

In this chapter, a first look at the effect of timing cuts on hadronic showers in the \geant simulation of the ILD detector has been done. In a first step, a perfect time resolution was assumed. It was shown that timing cuts affect the calorimeter response (up to 20\% for 5 GeV kaon) and energy resolution (up to 20-30\% relative to 100 ns for 5 GeV kaon). It has been shown that most of the shower energy is deposited within 10-15 ns and that timing cuts are removing mostly outer hits from the hadronic shower, thus reducing the width of the shower without affecting much the energy resolution. This could be used as a starting point in order to improve clustering and pattern recognition in PandoraPFA. The effect of timing cuts on the energy resolution is mostly an increase of the constant term. This has been investigated and it is understood that the timing cut introduces a bias on hadronic shower by cutting away late hits mostly coming from the hadronic part of the shower, thus reducing the hadronic response and enhancing the electromagnetic response. Timing cuts act as a non-compensation effect in the response to hadronic showers thus degrading the energy resolution.

Finally, a more realistic study has been done, assuming a time resolution of the AHCAL front-end electronics between 0.4 to 8 ns. It shows that a time resolution in the order of 1 ns and below has little impact on the response and resolution of the calorimeter. The energy resolution is degraded up to 6-8\% (absolute) for a tight timing cut but can reduce the shower width by 40\%. Such reduction of the shower width by applying timing cuts could be used to improve the pattern recognition. As for the currently achieved time resolution of 8 ns, it is not ideal but could still, to a certain level, provide information that could be used to improve shower separation. However, this study shows the effects of different time resolutions but does not give the answer what time resolution is needed as it depends mostly on the aim for the use of timing cuts. A time resolution of 8 ns is perfectly fine if it is only needed to separate bunch-crossings (around 200 ns apart at the ILC). However, a better time resolution would give access to a powerful tool that could be used to separate nearby showers.

In the future, analysis using time information should be performed in order to evaluate the potential of timing and the needed time resolution in order to help in the separation of overlapping nearby hadronic showers, the pattern recognition and as well improve the calorimeter energy resolution by software compensation.
