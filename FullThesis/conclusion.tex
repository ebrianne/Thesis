%!TEX root = main.tex
\cleardoublepage
\phantomsection
\addcontentsline{toc}{chapter}{Conclusion}
\chapter*{Conclusion}

The International Linear Collider (ILC) is a future linear electron-positron collider experiment. It will require an unprecedented jet energy resolution to reach the goals in the precision measurement of the Standard Model parameters. In order to achieve the jet energy resolution in the range of 3-4\%, the Particle Flow Algorithms (PFAs) concept combines the tracking and calorimeter measurements into the best jet energy measurement, by measurement each individual particle in a jet and using the best sub-detector resolution to measure the energy. PFAs need to correctly associate energy depositions in the calorimeters and tracks. This needs an unprecedented spatial resolution which can be achieved with high granular calorimeters.

The CALICE collaboration develops such high granularity calorimeter prototypes. One of the design consists of plastic scintillator tiles of $3\times3$ cm$^2$ area, read out by Silicon Photomultipliers (SiPMs). Several prototypes have been built and tested in various beams as a proof-of-concept. Nowadays, the focus of such calorimeters is the scalability to a full linear collider detector by integrating the front-end electronics onto the active layers.

This thesis presents the first timing analysis of a large scale analog hadron calorimeter based on scintillator-SiPM technology. In the testbeam campaign at CERN in July 2015, the CALICE AHCAL technological calorimeter prototype was operated in muon, electron and pion beams in an energy range up to 90 GeV. The main challenges in this analysis are the enormous number of channels to be calibrated in energy and time, the understanding of the different features of the front-end electronics and the contamination of beam events with multi-particles of different types and energy.

Firstly, the thesis here presents the commissioning of the AHCAL boards (HBU) that were used in the testbeam at CERN. Around 60 SPIROC chips have been tested manually with a yield of 84\%. It takes around 10 minutes per chip to be tested. The commissioning of an HBU is done is several steps, in total, 3 EBU and 24 HBU boards have been commissioned. Around 7-8 hours are needed to perform the commissioning for old generation boards due to the high range of LED voltage needed to calibrate the SiPM gain. New generation boards and improvements in SiPM quality reduce the commissioning time under 1 hour.

Secondly, the energy scale calibration of the full testbeam prototype has been presented.
