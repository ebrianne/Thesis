%!TEX root = ../main.tex
\part{Theory}

\chapter{Particle Physics: Theory}
\label{chap:Theory}

In this chapter, the theoretical foundations for this thesis are described. A brief overview of the Standard Model of particle physics (\acrshort{sm}) is given in section \ref{sec:SM}. This is followed by a brief description in section \ref{sec:HiggsTheo} of the Higgs mechanism and the particle itself, which is one of the corner of the Standard Model. Finally, open-questions of the Standard Model with possible solutions will be discussed in section \ref{sec:BeyondSM}.

\section{The Standard Model of Particle Physics}
\label{sec:SM}

It is difficult to say when particle physics is born. It has its origins in the discovery of the electron by Thomson \cite{JJThomson:1897} in 1897 and followed by the scattering experiment of Rutherford \cite{Rutherford:1911} in 1909 along with the discovery of radioactivity. This gave rise to the classical model of the atom by Niels Bohr. Following further experiments, it was realized that atoms themselves have an internal structure, and thus leading to the discovery of the nucleons, the proton and the neutron. With energies going higher and higher, more and more particles were discovered. In parallel, a theory that tried to understand radiation and the forces emerged. Finally, it was understood that most of the new discovered particles were composite objects. All sub-atomic particles could be reduced to a small set of \textit{elementary particles} which are known nowadays as the smallest constituents of our universe.

\subsection{The Standard Model framework}

In nature, conservation laws are generally associated with symmetries. This is one fundamental base of particle physics and all interactions between particles can be described by symmetries. The mathematical framework of the Standard Model is the Quantum Field Theory (\acrshort{qft}). The electromagnetic interaction is described by Quantum electrodynamics (\acrshort{qed}), the strong interaction is described by Quantum chromodynamics (\acrshort{qcd}) and the weak interaction is described by the weak theory.

In QFT, each elementary particle is described by a field $\Phi$. To put it simply, a field is a property of space. In QFT, there is no particles, only fields. Particles are a manifestation of the excitation of the fields. The interactions (kinematics and dynamics) of the theory are given by the \textit{Lagrangian density} $\mathcal{L}$ as a function of the field $\Phi(x)$ and its derivatives $\partial_{\mu}\Phi(x)$.

Symmetries are described by requiring the Lagrangian to be invariant under certain transformations. The Standard Model theory is based on gauge group theory and it can be formulated as
\begin{equation}
  SU(3)_{C} \times SU(2)_{L} \times U(1)_{Y}
\end{equation}
where $SU(3)_{C}$ is for QCD and $SU(2)_{L} \times U(1)_{Y}$ for the electroweak theory. The electroweak theory unifies the electromagnetic and weak interactions and is known as the \textit{Glashow-Weinberg-Salam} theory (GWS). These gauge groups are described by the Yang-Mills theory \cite{Yang:1954ek}.

\subsection{Elementary matter particles}

The Standard Model of particle physics is a theory that describes all elementary particles and their interactions. The SM is divided in different classes of elementary matter particles: the \textit{quarks} and the \textit{leptons} otherwise known as \textit{fermions}. All elementary matter particles have a spin of 1/2. Fermions are point-like particles without any internal structure. The interactions between particles are mediated by \textit{gauge bosons} (see section \ref{subsec:Mediators}).

In the Standard Model, there are six \textit{flavors} of quarks and leptons that are divided into three generations according to their masses and their quantum numbers: the \textit{charge} ($Q$), the \textit{spin} ($S$), the \textit{color charge} and the \textit{flavor}. The quarks are divided into two categories: up-type ($u$, $c$, $t$) and down-type quarks ($d$, $s$, $b$) which denote their flavor. Only quarks and gluons (see section \ref{subsec:Mediators}) carry a color charge ($r$, $g$, $b$).

The leptons are divided into charged leptons ($e^-$, $\mu^-$, $\tau^-$) and neutrinos ($\nu_e$, $\nu_{\mu}$, $\nu_{\tau}$). Each lepton has an antiparticle of the same mass and quantum numbers except that they have a opposite sign of charge.

Table \ref{table:Fermions} sums up the fundamental fermions and their properties described in the Standard Model.

\begin{table}[htb!]
  \centering
  \caption{Elementary fermions and their properties in the Standard Model. The masses are from \cite{Patrignani:2016xqp}. The quark masses depend on the calculation scheme.}
  \label{table:Fermions}
  \begin{tabular}{@{}lllll@{}} \toprule
    & \multicolumn{2}{c}{Leptons} & \multicolumn{2}{c}{Quarks} \\ \cmidrule(r){2-5}
    & charged & neutrino & up-type & down-type \\ \midrule
    Charge [$e$] & $\pm$ 1 & 0 & +2/3 & -1/3 \\
    Interact weakly & yes & yes & yes & yes \\
    Interact strongly & no & no & yes & yes \\ \midrule
    1$^{st}$ generation & $e$ (electron) & $v_e$ & $u$ (up) & $d$ (down) \\
    Mass [GeV] & $5.1 \times 10^{-4}$ & $< 2 \times 10^{-9}$ & $\sim 2.2 \times 10^{-3}$ & $\sim 4.7 \times 10^{-3}$\\ \midrule
    2$^{nd}$ generation & $\mu$ (muon) & $v_{\mu}$ & $c$ (charm) & $s$ (strange) \\
    Mass [GeV] & 0.105 & $< 0.19 \times 10^{-3}$ & $\sim$1.28 & $\sim 9.6 \times 10^{-2}$\\ \midrule
    3$^{rd}$ generation & $\tau$ (tau) & $v_{\tau}$ & $t$ (top) & $b$ (bottom) \\
    Mass [GeV] & 1.78 & $< 18.2 \times 10^{-3}$ & $\sim$173.1 & $\sim$4.18\\
    \bottomrule
  \end{tabular}
\end{table}

\subsection{Fundamental Forces and Mediators}
\label{subsec:Mediators}

In the Standard Model, the fundamental forces and interactions are known as:
\begin{itemize}
  \item The \textit{strong force} which is responsible for the interactions of color charged particles.
  \item The \textit{electromagnetic} force which is responsible for the interactions between charged particles.
  \item The \textit{weak force} which is responsible for flavor changing interactions e.g. $\beta$-decay.
  \item The \textit{gravity} which describes interactions between macroscopic and massive objects.
\end{itemize}

Forces are mediated by \textit{gauge bosons} which are a kind of elementary particle of spin 1. The gauge bosons interact with the particles by transferring a discrete amount of energy.

Firstly, the strong force, represented in $SU(3)_{C}$. The mediators of the strong force are 8 massless gluons ($g$) which carry a mixture of color and anti-color charges \cite{Griffiths:343277}.

Secondly, the electromagnetic force is mediated by another massless boson, the photon ($\gamma$).

Thirdly, the weak interaction is mediated by three gauge bosons: the electrically charged $W^+$ and $W^-$ bosons and the neutral $Z^0$ boson. In the Standard Model, the electromagnetic and weak force are unified in the electroweak theory described by the $SU(2)_{L} \times U(1)_{Y}$ symmetry. It is composed of the triplet $W_{\mu}^1, W_{\mu}^2, W_{\mu}^3$ in $SU(2)_{L}$ and a neutral field $B_{\mu}$ in $U(1)$. The mixture of these fields gives rise to the four bosons ($W^{+/-}, Z^0, \gamma$).

Finally, the gravity is not integrated yet into the Standard Model. It appears as the weakest force among all forces and it becomes only relevant for macroscopic objects or at very large distances. It is well described by the \textit{general relativity} theory \cite{Einstein:1905ve} and it is important for astrophysics and cosmology. The gravitational force can be neglected for microscopic objects comparing to the other forces. Table \ref{table:Bosons} sums up the bosons with their properties described in the Standard Model.

\begin{table}[htb!]
  \centering
  \caption{Bosons in the Standard Model. The masses are from \cite{Patrignani:2016xqp}. The photon and gluon are assumed to be massless. Gravitation is not considered}
  \label{table:Bosons}
  \begin{tabular}{@{}lllll@{}} \toprule
    Interaction & Particle & Mass [GeV] & Charge [$e$] & Spin\\
    \midrule
    Strong & $g \times 8$ & 0 & 0 & 1\\
    Electromagnetic & $\gamma$ & 0 & 0 & 1\\
    \multirow{2}{*}{Weak} & $W^{+/-}$ & 80.385 $\pm$ 0.015 & $\pm$ 1 & \multirow{2}{*}{1}\\
    & $Z^0$ & 91.1876 $\pm$ 0.0021 & 0 &\\
    Higgs & $H$ & 125.09 $\pm$ 0.24 & 0 & 0\\
    \bottomrule
  \end{tabular}
\end{table}

The last particle to describe is the Higgs Boson ($H$) which is an elementary particle of spin 0. This boson is one of the cornerstone of the Standard Model. Its properties are shown in table \ref{table:Bosons}.

The principle of gauge invariance (see section \ref{sec:HiggsTheo}) requires that gauge bosons are massless which is fulfilled by the photon and gluon but not by the $W^{+/-}$ and $Z^0$ bosons (see table \ref{table:Bosons}). It means that the electroweak gauge symmetry must be broken to generate masses for the heavy bosons. This is called the \textit{Electroweak Symmetry Breaking} (EWSB) mechanism. It gives rise to the \textit{Higgs mechanism}, which allows all other massive elementary particles to interact with the Higgs field to acquire mass. This mechanism is explained in section \ref{subsec:HiggsMecha}.

The Higgs theory \cite{Higgs:1964pj, Englert:1964et} has been formalized by P. Higgs, F. Englert and R. Brout$^\dagger$\footnote{Diseased in 2011} in 1964. The predicted Higgs particle has been discovered in 2012 at the Large Hadron Collider \cite{Aad:2012tfa, Chatrchyan:2012xdj} with a mass of about 125 GeV and is compatible with the Standard Model.

\section{The Higgs Boson}
\label{sec:HiggsTheo}

\subsection{Gauge invariance in the electroweak sector}

In the Standard Model, the Lagrangian $\mathcal{L}_{D}$, also known as the \textit{Dirac Lagrangian}, describes a free particle with a spin of 1/2 such as the electron \cite{Griffiths:343277}
\begin{equation}
  \mathcal{L}_{D} = i(\hbar c)\overline{\rm \Psi}\gamma^{\mu}\partial_{\mu}\Psi - (mc^2)\overline{\rm \Psi}\Psi
\end{equation}
where $\Psi$ a spinor field, $\gamma^{\mu}$ the Dirac $\gamma$-matrices (four matrices) \cite{Peskin:1995ev} and $m$ the mass of the particle. This Lagrangian is invariant under the \textit{global} gauge transformation $\Psi \rightarrow e^{i\theta}\Psi$ with $\theta$ being constant (equivalent to a simple phase rotation). However, it is not invariant under a \textit{local} gauge transformation where the phase factor $\theta$ is a function of space-time $\theta(x)$.

The local gauge invariance can be restored by adding a new vector field $A_{\mu}$ which follows the local gauge transformation $A_{\mu} \rightarrow A_{\mu} + \partial_{\mu}\lambda$ and by changing the partial derivative $\partial_{\mu}$ with the \textit{covariant derivative} $\mathcal{D}_{\mu} = \partial_{\mu} + i\frac{q}{\hbar c}A_{\mu}$. The local gauge invariant Lagrangian for a free particle of spin 1/2 becomes \cite{Griffiths:343277}
\begin{equation}
  \mathcal{L}_{QED} = i(\hbar c)\overline{\rm \Psi}\gamma^{\mu}\mathcal{D}_{\mu}\Psi - (mc^2)\overline{\rm \Psi}\Psi + (q\overline{\rm \Psi}\gamma^{\mu}\Psi)A_{\mu} - \frac{1}{4}F^{\mu\nu}F_{\mu\nu}
\end{equation}
where $F^{\mu\nu} \equiv \partial^{\mu}A^{\nu} - \partial^{\nu}A^{\mu}$ is the field strength tensor. The term $(q\overline{\rm \Psi}\gamma^{\mu}\Psi)A_{\mu}$ describes the interaction between the fermionic spinor field and the new field, which can be identified as the massless photon.

$\mathcal{L}_{QED}$ is invariant under $U(1)$ assuming that the new field is massless ($m_{A} = 0$). Otherwise, a mass term $m_{A}^2A^{\mu}A_{\mu}$ needs to be included, which breaks the local gauge symmetry. Evidences \cite{Lakes:1998mi, Chibisov:1976mm, Williams:1971ms} have shown that the photon is massless, and therefore $\mathcal{L}_{QED}$ describes well the fermions fields interacting with the photon field, known as Quantum Electrodynamics (\acrshort{qed}).

A similar approach was done by \textit{Yang-Mills} for the weak interaction. It requires the Lagrangian to be invariant under $SU(2)_{L}$ local gauge transformation. This is done by changing the partial derivative with the according \textit{covariant derivative} to satisfy local gauge invariance under $SU(2)_{L}$. The resulting Lagrangian is in the form \cite{Griffiths:343277}
\begin{equation}
  \mathcal{L}_{YM} = i(\hbar c)\overline{\rm \Psi}\gamma^{\mu}\mathcal{D}_{\mu}\Psi - (mc^2)\overline{\rm \Psi}\Psi + (q\overline{\rm \Psi}\gamma^{\mu}\mathbf{T}\Psi) \cdot \mathbf{A_{\mu}} - \frac{1}{4}\mathbf{F^{\mu\nu}} \cdot \mathbf{F_{\mu\nu}}, \quad \Psi = \begin{pmatrix} \Psi_1 \\ \Psi_2 \end{pmatrix}
\end{equation}
with $\Psi_1$ and $\Psi_2$ are four-component field spinors, $\mathcal{D}_{\mu} \equiv \partial_{\mu} + i\frac{q}{\hbar c}\mathbf{T} \cdot \mathbf{A_{\mu}}$ is the covariant derivative. $\mathbf{T}$ refers to the Pauli matrices and $\mathbf{A_{\mu}} = (A_{\mu}^1, A_{\mu}^2, A_{\mu}^3)$ is the new gauge field. This Lagrangian satisfies local gauge symmetry in $SU(2)_{L}$ only if the new gauge bosons are massless.

However, experimental evidences \cite{Rubbia:1983pta} show that the weak bosons $W^{+/-}$ and $Z^0$ have large masses (see table \ref{table:Bosons}). Therefore, the local gauge symmetry must be broken to include their mass terms in the Lagrangian $\mathcal{L}_{YM}$. A new concept is needed in order to restore the local gauge symmetry.

\subsection{The Higgs mechanism}
\label{subsec:HiggsMecha}

The electroweak gauge group $SU(2)_{L} \times U(1)_{Y}$ does not allow for fermions and bosons to be massive in order to conserve the local symmetry. A new framework is needed in order to give mass to the $W^{+/-}$ and $Z^0$ bosons while conserving the local gauge symmetry. The electroweak symmetry breaking (EWSB) provides such framework. It postulates that a self-interacting complex scalar field exists (see section \ref{sec:HiggsSM}) predicting a scalar particle known as the Higgs boson.

In the Standard Model, the Lagrangian for a scalar field $\Phi$ of mass $m$ with spin 0 is written as \cite{Griffiths:343277}
\begin{equation}
  \mathcal{L} = \frac{1}{2}(\partial_{\mu}\Phi)^{\dagger}(\partial^{\mu}\Phi) - \frac{1}{2}\left(\frac{mc}{\hbar}\right)^2\Phi^2
\end{equation}
and can be re-written (from classical Lagrangian) as
\begin{equation} \label{eq:HiggsLag}
  \mathcal{L} = \frac{1}{2}(\partial_{\mu}\Phi)^{\dagger}(\partial^{\mu}\Phi) - V(\Phi)
\end{equation}
where $V(\Phi)$ is the potential term. The potential can be written as:
\begin{equation} \label{eq:HiggsPotential}
  \begin{aligned}
    V(\Phi) & = \mu^2(\Phi^*\Phi) + \lambda(\Phi^*\Phi)^2 \\
    & = \mu^2|\Phi|^2 + \lambda|\Phi|^4
  \end{aligned}
\end{equation}
where $\mu$ and $\lambda > 0$ are real constants. If $\mu^2 > 0$, the vacuum is at zero and the Lagrangian is symmetric. It describes a particle of mass $\mu$ with a self-interaction term. However, if $\mu^2 < 0$, this induces that the potential has a non-zero field value of $v = \sqrt{-\mu^2/\lambda}$. To circumvent this, we can introduce a field centered at the vacuum $\eta = \Phi - v$. Rewriting the Lagrangian in this case yields
\begin{equation}
  \mathcal{L} = \frac{1}{2}(\partial_{\mu}\eta)^{\dagger}(\partial^{\mu}\eta) - \lambda v^2\eta^2 + \text{cubic and quadratic terms}
\end{equation}
This Lagrangian describes the kinematics for a scalar particle of mass $m_{\eta} = \sqrt{2\lambda}v$, the Higgs mass. This describes the principle for the Higgs mechanism for an Abelian\footnote{More commonly a commutative group i.e $a \cdot b = b \cdot a$} group ($U(1)$).

\subsection{Interpretation in the SM}
\label{sec:HiggsSM}

It can be generalized to a non-Abelian\footnote{The counterpart of Abelian groups where commutation is not applicable i.e $a \cdot b \neq b \cdot a$} group ($SU(2)_{L} \times U(1)_{Y}$). In $SU(2)$, the Higgs field is a complex doublet defined as
\begin{equation}
\Phi = \begin{pmatrix} \Phi^+ \\ \Phi^0 \end{pmatrix}= \frac{1}{\sqrt{2}} \begin{pmatrix} \Phi_1+i\Phi_2 \\ \Phi_3+i\Phi_4 \end{pmatrix}
\end{equation}
where $\Phi^+$ is the charged complex component and $\Phi^0$ the neutral component. Following the mechanism explain in section \ref{subsec:HiggsMecha}, if $\mu^2 < 0$ in equation \ref{eq:HiggsPotential}, the symmetry is spontaneously broken such as the neutral component of the Higgs field take a non-zero vacuum expectation value. At the ground state, the Higgs field is of the form
\begin{equation}
\Phi_0 = \frac{1}{\sqrt{2}}\begin{pmatrix} 0 \\ v \end{pmatrix}
\end{equation}
The breaking of the symmetry "eats" three of the four component of the Higgs field.

The Higgs Lagrangian is then becoming
\begin{equation}
  \mathcal{L}_{Higgs} = \underbrace{\frac{1}{2}(\mathcal{D}_{\mu}\Phi)^{\dagger}(\mathcal{D}^{\mu}\Phi)}_{\text{Gauge boson mass term}} - \underbrace{V(\Phi)}_{\text{Higgs mass + self-interaction}}
\end{equation}
where the covariant derivative operator of $SU(2)_{L} \times U(1)_{Y}$ is $\mathcal{D}_{\mu} \equiv \partial_{\mu} + ig\mathbf{T} \cdot \mathbf{W_{\mu}} + ig^{\prime} B_{\mu}Y$ where $g$ and $g^{\prime}$ are the gauge couplings, $\mathbf{W_{\mu}}$ and $B_{\mu}$ are the gauge fields.

This leads to the $W^{+/-}$ and $Z^0$ to acquire mass through interaction with the Higgs such as
\begin{equation}
  \begin{aligned}
    m_{W} &= g\frac{v}{2}\\
    m_{Z} &= \sqrt{g^2 + g^{\prime 2}}\frac{v}{2}
  \end{aligned}
\end{equation}

The couplings of the gauge bosons with the Higgs are proportional to the square of the boson mass given by
\begin{equation}
  g_{HVV} \propto \frac{m_V^2}{v}
\end{equation}

In quantum mechanics, fermions can be left-handed or right-handed (chirality). It is important to make the distinction as the Standard Model treats the chirality differently because only left-handed fermions (right-handed anti-fermions) interact with the W boson. In this case, left-handed fermions are doublets and right-handed are singlets. Right-handed neutrinos have never been observed and are not included here. Both type of fermions interact with the Higgs field and acquire masses through the \textit{Yukawa interactions}. The Yukawa Lagrangian is given by \cite{ILC_TDR_Vol2}
\begin{equation}
  \mathcal{L}_{Yukawa} = - y^u_{ij}\bar{u}_{Ri}\widetilde{\Phi}^{\dagger}Q_{Lj} - y^d_{ij}\bar{d}_{Ri}\Phi^{\dagger}Q_{Lj} - y^l_{ij}\bar{l}_{Ri}\Phi^{\dagger}L_{Lj} + \text{h.c}.
\end{equation}
where $Q_{Lj}$ are left-handed quarks, $\bar{u}_{Ri}$ are right-handed up-type quarks, $\bar{d}_{Ri}$ are right-handed down-type quarks, $L_{Lj}$ are the left-handed leptons and $\bar{l}_{Ri}$ are the right-handed leptons. The term $\widetilde{\Phi} = iT_2\Phi^{*}$ is the Higgs conjugate. The terms $y^u$, $y^d$ and $y^l$ are the Yukawa coupling matrices for up-type quarks, down-type quarks and charged leptons respectively.

All the couplings of fermions with the Higgs can be predicted and are proportional to the fermion mass and $v$ given by
\begin{equation}
  g_{Hff} \propto \frac{m_f}{v}
\end{equation}

The measurement of the Higgs mass is one of the important free parameters of the Standard Model. A precise measurement of the mass is necessary to validate the Standard Model. The Higgs has a large branching ratio, $\sim$58\%, decaying to a pair of bottom quarks. However, in hadron colliders such as the Large Hadron Collider (\acrshort{lhc}), this channel has a very poor signal compared to the background and a poor mass resolution due to large QCD background and thus makes it very challenging for a precise mass measurement.

Lepton colliders (see chapter \ref{chap:FutureColliders}) can offer a precise measurement of the Higgs Boson properties for example by exploiting the $H \rightarrow b\bar{b}$ channel. This implies the measurement of a final hadronic state. The specifics of hadrons are explained in the following section.

\section{Hadron physics}
\label{sec:Hadrons}

Quarks are fundamental particles of the Standard Model govern by the strong interaction. Bound states of quarks forming colorless particles are commonly called \textit{hadrons}. Hadrons are categorized as either \textit{mesons}, which are composed of a pair of quark-antiquark, or \textit{baryons} which are composed of three quarks. The theory of the strong interaction in QFT is known as quantum chromodynamics (QCD). QCD describes the strong interaction under the gauge symmetry group $SU(3)$. Massless bosons known as \textit{gluons} are the mediators of the strong interaction and they can self-interact (on the contrary in QED, the photon has no self-interaction). They carry a mixture of color and anti-color charges creating nine different combinations corresponding to the generators of $SU(3)$.

Quarks have been proven experimentally but no \textit{free} quarks have been observed. This is explained by the \textit{color confinement} in which color charged particles cannot be isolated, and thus quarks cannot be observed directly.

An important aspect of QCD is the \textit{asymptotic freedom}. It means that the strong coupling constant $\alpha_s$ decreases and become small at high energy scales, i.e. at short distances. Asymptotic freedom has the effect that quarks inside a hadron are considered free and not tightly bounded.

In particle physics, experiments observe only hadrons and not quarks. Quarks turn into colorless hadrons through a process called the \textit{hadronization}. This process can only be described phenomenologically. There are three main axioms: the independent hadronization, the string hadronization \cite{Artru1988} and the cluster hadronization \cite{Webber:1983if} but many variant models exist. None of them can be chosen more correct than the other, but they aim to represent the data correctly and to have some predictive power.

In a simple system of a di-jet production in a $e^+e^-$ annihilation, the $q\bar{q}$ partons move away from the common production vertex. The partons are bounded by a \textit{tube} or \textit{string} that confines both objects. At short distances ($<$1 fm), the $q\bar{q}$ pair is considered free. As they are colored particles they can radiate gluons. Since gluons carry color charges, they can also emit radiation. This process is known as \textit{parton showering} \cite{Feynman:1969wa}.

Once the distance become large enough, the energy stored in the string will be sufficient to produce a $q\bar{q}'$ pair. This process continues and creates a cascade of new $q\bar{q}$ pairs until the energy is low enough to hadronize the quarks into colorless hadrons. Unstable hadrons decay further into more stable particles that are observed in the detector. This creates a collimated cascade of particles known as a \textit{jet}.

\section{Beyond the Standard Model}
\label{sec:BeyondSM}

The Standard Model succeeds in describing many experimental results. The current Standard Model is composed of 19 free parameters which are:

\begin{itemize}
  \item The three leptons masses ($m_e$, $m_{\mu}$, $m_{\tau}$)
  \item The six quark masses ($m_u$, $m_d$, $m_c$, $m_s$, $m_b$, $m_t$)
  \item The strong gauge coupling constant $g_s$, the electroweak gauge coupling constants $g$ and $g^{\prime}$
  \item The vacuum expectation value $v$ and the Higgs mass $m_H$
  \item The weak mixing angles ($\theta_{12}$, $\theta_{23}$, $\theta_{13}$) and the CP phase $\delta$ for the flavor-changing weak decays \cite{Kobayashi:1973pt}
\end{itemize}

These parameters have their numerical values determined from experimental data. Nevertheless, there are still open questions that need to be answered, suggesting that the SM is maybe a \textit{low energy} approximation of a higher theory.

\subsection{Open questions of the Standard Model}

\subsubsection*{Gravity}

Gravity is a fundamental force in nature and is explained by the theory of general relativity. The Standard Model does not include gravity and the inclusion of the general relativity into the SM has not been successful so far.

\subsubsection*{Dark Matter}

Astrophysics and Cosmology have given evidences for the existence of invisible matter, e.g. from galaxies observations \cite{Battistelli:2017zrp}, the Cosmological Microwave Background (CMB) \cite{Giesen:2012rp}. This invisible matter is known as the \textit{dark matter}. As these particles have not been found so far, the interaction of dark matter with EM force and strong force have been ruled out. It is supposed that the couplings of dark matter is only through the weak interaction. New physics beyond the Standard Model is needed to explain the nature of these particles.

\subsubsection*{Neutrino masses}

No neutrino masses are present in the Standard Model. As there is no right-handed neutrino, neutrinos cannot acquire mass through Yukawa coupling. However, experiments on neutrino oscillations \cite{Dore:2008dp} shows that neutrinos have a mass. The mechanism by which neutrinos gain mass is so far unclear.

\subsubsection*{Hierarchy problem}

The Higgs mass of around 125 GeV is a theoretical challenge known as the hierarchy problem. In the SM, the mass of the Higgs gets loop corrections from fermion, gauge boson and the Higgs itself \cite{Vieira:2012ex}. If the Standard Model would be valid up to the Plank scale $\Lambda \sim 10^{19}$ GeV, these corrections would make the mass of the Higgs large. To keep the Higgs mass at the electroweak scale $\sim$100 GeV, it would require a very fine-tuning i.e cancellations between the loop corrections and the bare mass. This is considered unnatural.

\subsubsection*{Asymmetry matter-antimatter}

The Standard Model predicts that matter and anti-matter should be in equal amounts in the observable universe. But it is well established that our universe is composed mostly of matter. This \textit{baryon asymmetry} is not explained in the Standard Model. \acrshort{cp} violation \cite{Ellis:1978hq} may be the answer for this asymmetry, it is present in the Standard Model but has only been observed for weak interactions in the quark sector. But the only condition of CP violation would not entirely explain the asymmetry \cite{Sakharov:1967dj}.\\[0.3cm]

Numerous theoreticians are extending the Standard Model to answer few of these questions. One approach is to add symmetries to the Lagrangian. This approach is called \textit{Supersymmetry} (\acrshort{susy}) which adds super-partners to the Standard Model particles differing of a spin of 1/2, i.e. each fermion has a boson super-partner and vice-versa.

\begin{center}
  \rule{0.5\textwidth}{.4pt}
\end{center}

The Standard Model is so far the most reliable theory of the elementary objects and the interactions that govern our world. The Higgs is one of the cornerstone of the theory, therefore, a very precise measurement of its properties such as mass and couplings are needed. This enables us to probe the Standard Model deeply and look for any deviations to the SM predictions that would suggest physics beyond the Standard Model. Measurements at hadrons colliders, like the \acrlong{lhc}, suggest that if any other particle exists, it would be at a multi-TeV scale. However, the effect of such particles on coupling measurements would be below the percent level. Currently, the LHC does not achieve the level of precision needed to distinguish theories beyond the Standard Model \cite{CMS:2013xfa}. A lepton collider would be the perfect tool to conduct precision measurements including the Higgs as shown in chapter \ref{chap:FutureColliders}.
