%!TEX root = ../main.tex
\chapter{Particle Physics: Theory}

In this chapter, the theoretical foundations for this thesis are described. A brief overview of the Standard Model of particle physics (SM) is given in section \ref{sec:SM}. This is followed by a brief description of the Higgs mechanism and the particle itself which is a corner of the Standard Model in section \ref{sec:HiggsTheo}. As this thesis is mainly focused on hadron calorimetry, a description of quark physics will be given in section \ref{sec:QCD}. And finally, some of the short-comings of the Standard Model and possible solutions will be discussed in section \ref{sec:BeyondSM}.

\section{The Standard Model of Particle Physics}
\label{sec:SM}

It is difficult to say when \textit{Particle Physics} is born. It has its origins with the discovery of the electron by Thomson \cite{JJThomson:1897} in 1897 and followed by the famous scattering experiment of Rutherford \cite{Rutherford:1911} in 1909. This gave rise to the \textit{classical} model of the atom by Niels Bohr and the discovery of the proton and neutron. Particle physics had the intention to simply answer the question of \textit{What is the matter made of?}. This planted the seeds for great ideas to come in particle physics.

\subsection{Elementary particles}

After numerous experiments between 1950's and 1970's, the Standard Model of particle physics was \textit{concluded} and described all known elementary particles and their interactions. The SM is divided in two classes of elementary particles: the \textit{quarks} and the \textit{leptons}. The interactions between particles are mediated by \textit{gauge bosons}. Leptons are point-like particles without any internal structure. On the contrary, protons and neutrons are not and are a bound state of quarks, ($uud$) forms a proton and ($udd$) forms a neutron.

In the SM, there are 6 \textit{flavors} of quarks and 6 leptons divided into three generations of \textit{fermions}. They are classified according to their quantum numbers. The \textit{charge} ($Q$), the \textit{spin} ($S$), the \textit{color charge} and \textit{flavor}. The quarks are divided into two categories: up-type ($u$, $c$, $t$) and down-type quark ($d$, $s$, $b$) which denotes their flavor. Only quarks carry a color charge ($r$, $g$, $b$) such as a bound state of quarks called a \textit{hadron} is colorless or white.

The leptons are also divided into charged leptons ($e^-$, $\mu^-$, $\tau^-$) and neutrinos ($\nu_e$, $\nu_{\mu}$, $\nu_{\tau}$). Each lepton has an antiparticle of the same mass but of opposite sign charge and quantum numbers. All elementary particles are fermions with a spin 1/2.

The table \ref{table:Fermions} sums up the fundamental fermions and their properties in the Standard Model.

\begin{table}[htb!]
  \centering
  \caption{Fermions in the Standard Model.}
  \label{table:Fermions}
  \begin{tabular}{|p{3cm}||cccc|}
    \hline
    Particle & Flavor & Charge ($Q$) & Color & Spin ($S$)\\
    \hline
    \multirow{2}{*}{Quarks} & $u$, $c$, $t$ & 2/3 & \multirow{2}{*}{r, g, b} & \multirow{2}{*}{1/2}\\
    & $d$, $s$, $b$ & -1/3 & & \\
    \hline
    \multirow{2}{*}{Leptons} & $e^-$, $\mu^-$, $\tau^-$ & -1 & \multirow{2}{*}{-} & \multirow{2}{*}{1/2}\\
    & $\nu_e$, $\nu_{\mu}$, $\nu_{\tau}$ & 0 & & \\
    \hline
  \end{tabular}
\end{table}

\subsection{Fundamental Forces and Mediators}

In the Standard Model, fundamental forces and their interactions are known as:
\begin{itemize}
  \item The \textit{strong force} which is responsible for the interactions with quarks and the binding of quark into matter.
  \item The \textit{electromagnetic} force which is responsible for the interactions between charged particles.
  \item The \textit{weak force} which is responsible for radioactive decay.
  \item \textit{Gravity} which describes interactions between masses.
\end{itemize}

Forces are mediated by \textit{gauge bosons} in the Standard Model. These gauge bosons interact with the particles by transferring a discrete amount of energy. The strong force is mediated by a massless particle, the gluon ($g$) which carries one unit of color and one unit of anticolor. The electromagnetic force is mediated by another massless boson, the photon ($\gamma$). And finally, the weak interaction is mediated by three gauge bosons: the electrically charged $W^+$ and $W^-$ bosons and the neutral $Z^0$ boson. In the Standard Model, the electromagnetic and weak force are unified under the electroweak theory. All gauge bosons have a spin 1. Gravity is somehow special, it is not integrated in the Standard Model, it is the weakest force among all forces and it becomes only relevant at very large distances. It is described well by the \textit{special relativity} theory \cite{Einstein:1905ve} and it is important for astrophysics and cosmology but negligible in particle physics. The table \ref{table:GaugeBosons} sums up the gauge bosons of the Standard Model.

\begin{table}[htb!]
  \centering
  \caption{Gauge bosons in the Standard Model. The masses for the $W^{+/-}$, $Z^0$ and $H$ are from \cite{Patrignani:2016xqp}. The Higgs boson is explained in section \ref{sec:HiggsTheo}.}
  \label{table:GaugeBosons}
  \begin{tabular}{|p{8cm}||cccc|}
    \hline
    Interaction & Boson & Mass [GeV] & Charge & Spin\\
    \hline
    Strong & $g$ & 0 & 0 & 1\\
    \hline
    Electromagnetic & $\gamma$ & 0 & 0 & 1\\
    \hline
    \multirow{2}{*}{Weak} & $W^{+/-}$ & 80.385 & $\pm$ 1 & \multirow{2}{*}{1}\\
    & $Z^0$ & 91.1876 & 0 &\\
    \hline
    \hline
    Electroweak Symmetry Breaking & Higgs ($H$) & 125.09 & 0 & 0\\
    \hline
  \end{tabular}
\end{table}

In nature, conservation law are generally associated to symmetries. This in one fundamental base of particle physics and all interactions can be described by symmetries. Symmetries are mathematical groups such as the \textit{Lagrangian}\footnote{The Lagrangian density $\mathcal{L}$ is a mathematical representation for the theory and is function of the field $\Phi_i$ and its derivatives $\partial_{\mu}\Phi_i$.} is invariant under certain transformations. The Standard Model theory is based on theory groups, quantum electrodynamics (QED) and quantum chromodynamics (QCD) otherwise known as the \textit{Glashow-Weinberg-Salam} theory (GWS) and can be formulated as
\begin{equation}
  SU(3)_{C} \times SU(2)_{L} \times U(1)_{Y}
\end{equation}
where $SU(3)_{C}$ represents the gauge group of chromodynamics (QCD), $SU(2)_{L} \times U(1)_{Y}$ represents the gauge group of electroweak theory. These gauge groups are described by the Yang-Mills theory \cite{Yang:1954ek}.

$SU(3)_{C}$ is composed of a color octet and color singlet. The singlet would in principle exists as a free particle and could be exchanged between two color singlets, implying a long-range force with strong coupling, this is obviously not the case in our world thus only 8 gluons are mediators \cite{Griffiths:343277}. The group $SU(2)_{L} \times U(1)_{Y}$ incorporates the four bosons ($W^{+/-}, Z^0, \gamma$) with the triplet $W_{\mu}^1, W_{\mu}^2, W_{\mu}^3$ in $SU(2)_{L}$ and a neutral $B_{\mu}$ in $U(1)$ such as:
\begin{equation}
  W_{\mu}^{+/-} = \frac{1}{\sqrt{2}}(W_{\mu}^1 \pm W_{\mu}^2)
\end{equation}
representing the $W^{+/-}$ bosons. The two neutral states $W_{\mu}^3$ and $B_{\mu}$ mix to produce the massless photon ($A_{\mu}$) and the massive $Z_{\mu}$ boson:
\begin{equation}
  \begin{aligned}
    A_{\mu} &= B_{\mu} cos(\theta_W) + W_{\mu}^3 sin(\theta_W)\\
    Z_{\mu} &= - B_{\mu} sin(\theta_W) + W_{\mu}^3 cos(\theta_W)
  \end{aligned}
\end{equation}
with $\theta_W$ the \textit{weak mixing} angle or \textit{Weinberg} angle. This angle is a single fundamental parameter and is defined as:
\begin{equation}
  g_W = \frac{g_e}{sin(\theta_W)}, \quad g_Z = \frac{g_e}{sin(\theta_W) cos(\theta_W)}
\end{equation}
with $g_W$, $g_Z$ and $g_e$ the weak and electromagnetic coupling constant. The masses of the $W^{+/-}$ and $Z^0$ bosons are related as $M_W = M_Z \cdot cos(\theta_W)$.

The last particle to describe is the Higgs Boson ($H$) of spin 0. This boson is the cornerstone of the Standard Model. The principle of gauge invariance requires that bosons are massless which is fulfilled for the photon and gluon but the $W^{+/-}$ and $Z^0$ bosons are massive particles. This means that $SU(2)_{L} \times U(1)_{Y}$ must be broken to generate masses for the heavy bosons. This is called the \textit{Electroweak Symmetry Breaking}. By introducing the Higgs boson, this breaks the $SU(2)_{L} \times U(1)_{Y}$ group by the \textit{Higgs mechanism} and generate masses for fermions and bosons. The theory \cite{Higgs:1964pj, Englert:1964et} has been formalized by P. Higgs, F. Englert and R. Brout$^\dagger$\footnote{Diseased in 2011} in 1964 and the particle has been discovered in 2012 at the LHC \cite{Aad:2012tfa, Chatrchyan:2012xdj}. They were awarded the Nobel Physics Prize in 2013 for their theory.

\section{The Higgs Boson}
\label{sec:HiggsTheo}

In the Standard Model, the Lagrangian $\mathcal{L}_{D}$, also known as the \textit{Dirac Lagrangian}, for free particle of spin 1/2 such as the electron is \cite{Griffiths:343277}
\begin{equation}
  \mathcal{L}_{D} = i(\hbar c)\overline{\rm \Psi}\gamma^{\mu}\partial_{\mu}\Psi - (mc^2)\overline{\rm \Psi}\Psi
\end{equation}
with $\Psi$ a spinor field, $\gamma^{\mu}$ the Dirac $\gamma$-matrices (4 matrices) \cite{Peskin:1995ev} and $m$ the mass of the particle. This Lagrangian is invariant under the \textit{global} gauge transformation $\Psi \rightarrow e^{i\theta}\Psi$ with $\theta$ being constant (equivalent to a simple phase rotation). However, this in not invariant under \textit{local} gauge transformation where the phase factor $\theta$ is function of space-time $\theta(x)$.

The local gauge invariance can be restored by adding a new field $A_{\mu}$ which follows the local gauge transformation $A_{\mu} \rightarrow A_{\mu} + \partial_{\mu}\lambda$ and changing the derivative $\partial_{\mu}$ by the \textit{covariant derivative} $\mathcal{D}_{\mu} = \partial_{\mu} + i\frac{q}{\hbar c}A_{\mu}$. The local gauge invariant Lagrangian for a free particle of spin 1/2 is then \cite{Griffiths:343277}
\begin{equation}
  \mathcal{L}_{QED} = i(\hbar c)\overline{\rm \Psi}\gamma^{\mu}\mathcal{D}_{\mu}\Psi - (mc^2)\overline{\rm \Psi}\Psi + (q\overline{\rm \Psi}\gamma^{\mu}\Psi)A_{\mu} - \frac{1}{4}F^{\mu\nu}F_{\mu\nu}
\end{equation}
with $F^{\mu\nu} \equiv \partial^{\mu}A^{\nu} - \partial^{\nu}A^{\mu}$ the photon field. The term $(q\overline{\rm \Phi}\gamma^{\mu}\Psi)A_{\mu}$ describes the interaction of the fermion with the photon. By introducing the new field $A_{\mu}$, a term $m^2A^{\mu}A_{\mu}$ should be present but breaks the local gauge symmetry thus the new field is massless (the photon). This Lagrangian describes well the fermions fields interacting with the photon field otherwise known as QED and is invariant under $U(1)$.

A similar approach was done by \textit{Yang-Mills} for the weak interaction, also for the strong interaction in QCD under $SU(3)$. It requires the Lagrangian to be invariant under $SU(2)_{L}$ local gauge transformation. This is done by changing the derivative with the according \textit{covariant derivative} to satisfy local gauge invariance under $SU(2)_{L}$. The resulting Lagrangian is in the form \cite{Griffiths:343277}
\begin{equation}
  \mathcal{L}_{YM} = i(\hbar c)\overline{\rm \Psi}\gamma^{\mu}\mathcal{D}_{\mu}\Psi - (mc^2)\overline{\rm \Psi}\Psi + (q\overline{\rm \Psi}\gamma^{\mu}\mathbf{T}\Psi) \cdot \mathbf{A_{\mu}} - \frac{1}{4}\mathbf{F^{\mu\nu}} \cdot \mathbf{F_{\mu\nu}}, \quad \Psi = \begin{pmatrix} \Psi_1 \\ \Psi_2 \end{pmatrix}
\end{equation}
with $\Psi_1$ and $\Psi_2$ four-component field spinors, $\mathcal{D}_{\mu} \equiv \partial_{\mu} + i\frac{q}{\hbar c}\mathbf{T} \cdot \mathbf{A_{\mu}}$ the covariant derivative. $\mathbf{T}$ are the Pauli matrices and $\mathbf{A_{\mu}} = (A_{\mu}^1, A_{\mu}^2, A_{\mu}^3)$ is the new gauge field. This Lagrangian satisfies local gauge symmetry in $SU(2)_{L}$ only if the new gauge bosons are massless! However, the weak bosons $W^{+/-}$ and $Z^0$ are proven to be heavy thus their mass term should be included in $\mathcal{L}_{YM}$ which would break the local symmetry. A similar problem arises for the fermions. To resolve this problem, the Higgs mechanism was beautifully created.

\subsection{The Higgs mechanism}
\label{subsec:HiggsMecha}

As shown previously, local symmetry does not allow for fermions and bosons to be massive. A new framework is needed in order to give mass to the particles and to conserve symmetry at the same time. This is where the Higgs mechanism and spontaneous symmetry breaking comes in. The Lagrangian for a scalar field $\Phi$ of mass $m$ with spin 0 is written as \cite{Griffiths:343277}
\begin{equation}
  \mathcal{L} = \frac{1}{2}(\partial_{\mu}\Phi)^{\dagger}(\partial^{\mu}\Phi) - \frac{1}{2}\left(\frac{mc}{\hbar}\right)^2\Phi^2
\end{equation}
and can be re-written (from classical Lagrangian) as
\begin{equation} \label{eq:HiggsLag}
  \mathcal{L} = \frac{1}{2}(\partial_{\mu}\Phi)^{\dagger}(\partial^{\mu}\Phi) - V(\Phi)
\end{equation}
with $V(\Phi)$ the potential term. Let's take a potential of the form:
\begin{equation}
  \begin{aligned}
    V(\Phi) & = \mu^2(\Phi^*\Phi) + \lambda(\Phi^*\Phi)^2 \\
    & = \mu^2|\Phi|^2 + \lambda|\Phi|^4
  \end{aligned}
\end{equation}
with $\mu$ and $\lambda > 0$ real constants. If $\mu^2 > 0$, the vacuum is at zero and the Lagrangian is symmetric. It describes a particle of mass $\mu$ with a self-interaction term. However, if $\mu^2 < 0$, this induces that the potential has a non-zero field value of $v = \sqrt{-\mu^2/\lambda}$. To circumvent this, we can introduce a field centered at the vacuum $\eta = \Phi - v$. Rewriting the Lagrangian in this case yields
\begin{equation}
  \mathcal{L} = \frac{1}{2}(\partial_{\mu}\eta)^{\dagger}(\partial^{\mu}\eta) - \lambda v^2\eta^2
\end{equation}
neglecting term higher than the 2$^{nd}$ order. This Lagrangian describes the kinematics for a scalar particle of mass $m_{\eta} = \sqrt{2\lambda}v$, the Higgs mass. This gives the principle for the Higgs mechanism for an Abelian\footnote{More commonly a commutative group i.e $a \cdot b = b \cdot a$} group ($U(1)$).

\subsection{Interpretation in the SM}

It can be generalized to non-Abelian\footnote{The counterpart of Abelian groups where commutation is not applicable i.e $a \cdot b \neq b \cdot a$} group ($SU(2)_{L} \times U(1)_{Y}$) by taking the field $\Phi \equiv \Phi_1+i\Phi_2$ as a complex scalar. The Higgs Lagrangian is then
\begin{equation}
  \mathcal{L}_{Higgs} = \underbrace{\frac{1}{2}(\mathcal{D}_{\mu}\Phi)^{\dagger}(\mathcal{D}^{\mu}\Phi)}_{\text{Gauge boson mass term}} - \underbrace{V(\Phi)}_{\text{Higgs mass + self-interaction}}
\end{equation}
with the covariant derivative operator of $SU(2)_{L} \times U(1)_{Y}$, $\mathcal{D}_{\mu} \equiv \partial_{\mu} + ig\mathbf{T} \cdot \mathbf{W_{\mu}} + ig^{\prime} B_{\mu}Y$ with $g$ and $g^{\prime}$ the gauge couplings and $\mathbf{W_{\mu}}$ and $B_{\mu}$ the gauge fields (representing the $W^{+/-}$, $Z^0$ bosons and the photon).

This leads to the $W^{+/-}$ and $Z^0$ to acquire mass through interaction with the Higgs
\begin{equation}
  \begin{aligned}
    m_{W} &= g\frac{v}{2}\\
    m_{Z} &= \sqrt{g^2 + g^{\prime 2}}\frac{v}{2}
  \end{aligned}
\end{equation}

The only piece of the puzzle is the interaction of the fermions with the Higgs field. As shown in section \ref{sec:HiggsTheo}, the term $m^2\overline{\rm \Psi}\Psi$ breaks local gauge symmetry and would imply that fermions are massless. The Higgs mechanism described in subsection \ref{subsec:HiggsMecha} can give fermions mass and keep the local gauge symmetry. This results in the Lagrangian of the form \cite{ILC_TDR_Vol2}
\begin{equation}
  \mathcal{L}_{Yukawa} = - y^u_{ij}\bar{u}_{Ri}\widetilde{\Phi}^{\dagger}Q_{Lj} - y^d_{ij}\bar{d}_{Ri}\Phi^{\dagger}Q_{Lj} - y^l_{ij}\bar{l}_{Ri}\Phi^{\dagger}L_{Lj} + \text{h.c}.
\end{equation}
with $Q_{Lj}$ left-handed quarks, $\bar{u}_{Ri}$ right-handed up-type quarks, $\bar{d}_{Ri}$ right-handed down-type quarks, $L_{Lj}$ the left-handed leptons and $\bar{l}_{Ri}$ the right-handed leptons. The term $\widetilde{\Phi} = iT_2\Phi^{*}$ is the Higgs conjugate. The terms $y^u$, $y^d$ and $y^l$ are the Yukawa coupling matrices for up-type quarks, down-type quarks and charged leptons respectively.

All the couplings of fermions with the Higgs can be predicted and are proportional to the fermion mass and $v$, $g_{Hff} \propto \frac{m_f}{v}$. But the fermion masses are not given in the Standard Model, they must be measured experimentally. The couplings of the gauge bosons with the Higgs are proportional to the square of the boson mass, $g_{HVV} \propto \frac{m_V^2}{v}$. The full Lagrangian in the Standard Model can be written as \cite{Woithe:2017lzd}
\begin{equation}
  \begin{aligned}
    \mathcal{L}_{SM} = &\underbrace{-\frac{1}{4}F^{\mu\nu}F_{\mu\nu}}_{\text{interaction particles}} + \underbrace{i\bar{\Psi}\mathcal{D}\Psi}_{\text{interactions between particles}} \\
    &+ \underbrace{\Psi_iy_{ij}\Psi_j\Phi}_{\text{Fermion mass}} \quad + \quad \underbrace{|D_{\mu}\Phi|^2}_{\text{Bosons mass}}\\
    &- \underbrace{V(\Phi)}_{\text{Higgs mass + self-interaction}} + \text{ h.c}
  \end{aligned}
\end{equation}

\section{Hadron physics}

Quarks are fundamental particles of the Standard Model govern by the strong interaction. Bound states of quarks forming particles have been observed since the middle of $20^{th}$ century. These particles are called \textit{hadrons}. The theory of the strong interaction in quantum field theory is known as quantum chromodynamics (QCD). QCD is similarly based on the concept of QED but in the $SU(3)$ gauge group. Therefore, 8 massless bosons known as \textit{gluons} are the mediators of the strong interaction and can also self-interact (on the contrary in QED, the photon has no self-interaction). Gluons carry a \textit{color charge} and anticolor charge at the same time. There are 3 different charges: red ($r$), blue ($b$), green ($g$). Thus this creates 9 possible combinations.

Quarks have been proven experimentally but no \textit{free} quarks have been observed. This is explained by \textit{color confinement} in which color charged particles must be colorless thus quarks cannot be isolated and cannot be observed directly.

An important aspect of QCD is \textit{asymptotic freedom}. This means that the strong coupling constant $\alpha_s$ decreases and become small at high energy scales, i.e. short distances. Asymptotic freedom has the effect that quarks inside a hadron a considered free and not tightly bounded.

In particle physics, experiments observe only hadrons and not quarks. Quarks turn into colorless hadrons though a process called \textit{hadronisation}. This process is not calculable in perturbation theory ($\alpha_s >> 1$) thus can only be described phenomenologically. Many different models are available: independent hadronisation, string hadronisation \cite{Artru1988} and cluster hadronisation \cite{Webber:1983if}.

When a quark-antiquark pair is created such as in a collision $e^+e^- \rightarrow q\bar{q}$, the quarks start to move away from each other. A short distances, the quarks are considered free. They can radiate gluons (similar to Bremsstrahlung for an electron) that in turn can radiate gluons. This process is known as \textit{parton showering}. Once the distance become large enough, quarks are bounded by a \textit{tube} or \textit{string} that confines both objects. At one point, the energy stored in the string will be sufficient to produce a new pair of quark-antiquark. This causes the color field to break into smaller fields of smaller energy. This process continues and creates a cascade of new $q\bar{q}$ pairs until the energy is low enough to combine the quark into hadrons. Unstable hadrons decay further into more stable particles. This creates a collimated cascade of particles known as a \textit{jet}.

\section{Beyond the Standard Model}
\label{sec:BeyondSM}

The Standard Model has been a wonderful theory to describe and predict many physical phenomena. The current Standard Model is composed of 19 free parameters which have no explanation from the theory and are chosen to match experimental data. Nevertheless, they are still open questions that need to be answered by the Standard Model and would suggest that maybe the SM is only a \textit{low energy} approximation of a higher theory.

\subsection{Short-comings of the SM}

\subsubsection*{Gravity}

Gravity is a fundamental force in nature and is explained by the theory of general relativity. The Standard Model does not include gravity and latest attempts to include general relativity have not been successful so far.

\subsubsection*{Dark Matter}

Astrophysics and Cosmology have given observations for the existence of invisible matter known as the \textit{dark matter}. As these particles have not been found so far, it suggests that the coupling of dark matter to electromagnetic and strong interactions is extremely low. New physics beyond the Standard Model is needed to explain the nature of these particles.

\subsubsection*{Neutrino masses}

No neutrino masses are present in the Standard Model as there is no right-handed neutrino thus it cannot acquire mass through Yukawa coupling. However, experiments on neutrino oscillations \cite{Dore:2008dp} shows that neutrinos have a mass (how extremely small it is). The mechanism by which neutrinos gain mass is so far unclear.

\subsubsection*{Hierarchy problem}

At high energies, the Higgs mass has large loop corrections \cite{Vieira:2012ex}. If the Standard Model is valid up to the Plank scale $\Lambda \sim 10^19$ GeV, these corrections become very large. Therefore, it is difficult to keep the Higgs mass at the electroweak scale $\sim$100 GeV and would require a very fine-tuning. This is considered unnatural.

\subsubsection*{Asymmetry matter-antimatter}

The Standard Model predicts that matter and anti-matter should be in equal amounts in the observable universe. But it is well established that our universe is composed mostly of matter. This \textit{baryon asymmetry} is not explained in the Standard Model. CP violation \cite{Ellis:1978hq} may be the answer for this asymmetry, it is present in the Standard Model but has only been observed for weak interactions in the quark sector. But the only condition of CP violation would not entirely explain the asymmetry and would require more conditions \cite{Sakharov:1967dj}.

Numerous theoretician are extending the Standard Model to answer few of these questions. One approach is to add symmetries to the Lagrangian. This approach is called \textit{Supersymmetry} (SUSY) which adds super-partners to the Standard Model particles differing of a spin of 1/2, i.e. each fermions has a boson super-partner and vice-versa.

\begin{center}
\rule{0.5\textwidth}{.4pt}
\end{center}

The Standard Model is so far the most reliable theory of the forces that govern our world. The Higgs is the cornerstone of the theory therefore a very precise measurement of its properties such as mass and couplings are needed. This enables us to probe the Standard Model deeply and look for any effects that would suggest physics beyond the Standard Model. Measurements at hadrons colliders like LHC does not achieve the level of precision needed to distinguished theories beyond the Standard Model. A lepton collider would be the perfect tool to conduct precision measurements including the Higgs.
