%!TEX root = ../main.tex
\chapter{Particle Physics: Theory}

In this chapter, the theoretical foundations for this thesis are described. A brief overview of the Standard Model of particle physics (SM) is given in section \ref{sec:SM}. This is followed by a brief description of the Higgs mechanism and the particle itself which is a corner of the Standard Model in section \ref{sec:HiggsTheo}. As this thesis is mainly focused on hadron calorimetry, a description of quark physics will be given in section \ref{sec:QCD}. And finally, some of the short-comings of the Standard Model and possible solutions will be discussed in section \ref{sec:BeyondSM}.

\section{The Standard Model of Particle Physics}
\label{sec:SM}

It is difficult to say when \textit{Particle Physics} is born. It has its origins with the discovery of the electron by Thomson \cite{JJThomson:1897} in 1897 and followed by the famous scattering experiment of Rutherford \cite{Rutherford:1911} in 1909. This gave rise to the \textit{classical} model of the atom by Niels Bohr and the discovery of the proton and neutron. Particle physics had the intention to simply answer the question of \textit{What is the matter made of?}. This planted the seeds for great ideas to come in particle physics.

\subsection{Elementary particles}

After numerous experiments between 1950's and 1970's, the Standard Model of particle physics was \textit{concluded} and described all known elementary particles and their interactions. The SM is divided in two classes of elementary particles: the \textit{quarks} and the \textit{leptons}. The interactions between particles are mediated by \textit{gauge bosons}. Leptons are point-like particles without any internal structure. On the contrary, protons and neutrons are not and are a bound state of quarks, ($uud$) forms a proton and ($udd$) forms a neutron.

In the SM, there are 6 \textit{flavors} of quarks and 6 leptons divided into three generations of \textit{fermions}. They are classified according to their quantum numbers. The \textit{charge} ($Q$), the \textit{spin} ($S$), the \textit{color charge} and \textit{flavor}. The quarks are divided into two categories: up-type ($u$, $c$, $t$) and down-type quark ($d$, $s$, $b$) which denotes their flavor. Only quarks carry a color charge ($r$, $g$, $b$) such as a bound state of quarks called a \textit{hadron} is colorless or white.

The leptons are also divided into charged leptons ($e^-$, $\mu^-$, $\tau^-$) and neutrinos ($\nu_e$, $\nu_{\mu}$, $\nu_{\tau}$). Each lepton has an antiparticle of the same mass but of opposite sign charge and quantum numbers. All elementary particles are fermions with a spin 1/2.

The table \ref{table:Fermions} sums up the fundamental fermions and their properties in the Standard Model.

\begin{table}[htb!]
  \centering
  \caption{Fermions in the Standard Model.}
  \label{table:Fermions}
  \begin{tabular}{|p{3cm}||cccc|}
    \hline
    Particle & Flavor & Charge ($Q$) & Color & Spin ($S$)\\
    \hline
    \multirow{2}{*}{Quarks} & $u$, $c$, $t$ & 2/3 & \multirow{2}{*}{r, g, b} & \multirow{2}{*}{1/2}\\
    & $d$, $s$, $b$ & -1/3 & & \\
    \hline
    \multirow{2}{*}{Leptons} & $e^-$, $\mu^-$, $\tau^-$ & -1 & \multirow{2}{*}{-} & \multirow{2}{*}{1/2}\\
    & $\nu_e$, $\nu_{\mu}$, $\nu_{\tau}$ & 0 & & \\
    \hline
  \end{tabular}
\end{table}

\subsection{Fundamental Forces and Mediators}

In the Standard Model, fundamental forces and their interactions are known as:
\begin{itemize}
  \item The \textit{strong force} which is responsible for the interactions with quarks and the binding of quark into matter.
  \item The \textit{electromagnetic} force which is responsible for the interactions between charged particles.
  \item The \textit{weak force} which is responsible for radioactive decay.
  \item \textit{Gravity} which describes interactions between masses.
\end{itemize}

Forces are mediated by \textit{gauge bosons} in the Standard Model. These gauge bosons interact with the particles by transferring a discrete amount of energy. The strong force is mediated by a massless particle, the gluon ($g$) which carries one unit of color and one unit of anticolor. The electromagnetic force is mediated by another massless boson, the photon ($\gamma$). And finally, the weak interaction is mediated by three gauge bosons: the electrically charged $W^+$ and $W^-$ bosons and the neutral $Z^0$ boson. In the Standard Model, the electromagnetic and weak force are unified under the electroweak theory. All gauge bosons have a spin 1. Gravity is somehow special, it is not integrated in the Standard Model, it is the weakest force among all forces and it becomes only relevant at very large distances. It is described well by the \textit{special relativity} theory \cite{Einstein:1905ve} and it is important for astrophysics and cosmology but negligible in particle physics. The table \ref{table:GaugeBosons} sums up the gauge bosons of the Standard Model.

\begin{table}[htb!]
  \centering
  \caption{Gauge bosons in the Standard Model. The masses for the $W^{+/-}$, $Z^0$ and $H$ are from \cite{Patrignani:2016xqp}. The Higgs boson is explained in section \ref{sec:HiggsTheo}.}
  \label{table:GaugeBosons}
  \begin{tabular}{|p{8cm}||cccc|}
    \hline
    Interaction & Boson & Mass [GeV] & Charge & Spin\\
    \hline
    Strong & $g$ & 0 & 0 & 1\\
    \hline
    Electromagnetic & $\gamma$ & 0 & 0 & 1\\
    \hline
    \multirow{2}{*}{Weak} & $W^{+/-}$ & 80.385 & $\pm$ 1 & \multirow{2}{*}{1}\\
    & $Z^0$ & 91.1876 & 0 &\\
    \hline
    \hline
    Electroweak Symmetry Breaking & Higgs ($H$) & 125.09 & 0 & 0\\
    \hline
  \end{tabular}
\end{table}

In nature, conservation law are generally associated to symmetries. This in one fundamental base of particle physics and all interactions can be described by symmetries. Symmetries are mathematical groups such as the \textit{Lagrangian}\footnote{The Lagrangian density $\mathcal{L}$ is a mathematical representation for the theory and is function of the field $\Phi_i$ and its derivatives $\partial_{\mu}\Phi_i$.} is invariant under certain transformations. The Standard Model theory is based on theory groups, quantum electrodynamics (QED) and quantum chromodynamics (QCD) otherwise known as the \textit{Glashow-Weinberg-Salam} theory (GWS) and can be formulated as
\begin{equation}
  SU(3)_{C} \times SU(2)_{L} \times U(1)_{Y}
\end{equation}
where $SU(3)_{C}$ represents the gauge group of chromodynamics (QCD), $SU(2)_{L} \times U(1)_{Y}$ represents the gauge group of electroweak theory. These gauge groups are described by the Yang-Mills theory \cite{Yang:1954ek}.

$SU(3)_{C}$ is composed of a color octet and color singlet. The singlet would in principle exists as a free particle and could be exchanged between two color singlets, implying a long-range force with strong coupling, this is obviously not the case in our world thus only 8 gluons are mediators \cite{Griffiths:343277}. The group $SU(2)_{L} \times U(1)_{Y}$ incorporates the four bosons ($W^{+/-}, Z^0, \gamma$) with the triplet $W_{\mu}^1, W_{\mu}^2, W_{\mu}^3$ in $SU(2)_{L}$ and a neutral $B_{\mu}$ in $U(1)$ such as:
\begin{equation}
  W_{\mu}^{+/-} = \frac{1}{\sqrt{2}}(W_{\mu}^1 \pm W_{\mu}^2)
\end{equation}
representing the $W^{+/-}$ bosons. The two neutral states $W_{\mu}^3$ and $B_{\mu}$ mix to produce the massless photon ($A_{\mu}$) and the massive $Z_{\mu}$ boson:
\begin{equation}
  \begin{aligned}
    A_{\mu} &= B_{\mu} cos(\theta_W) + W_{\mu}^3 sin(\theta_W)\\
    Z_{\mu} &= - B_{\mu} sin(\theta_W) + W_{\mu}^3 cos(\theta_W)
  \end{aligned}
\end{equation}
with $\theta_W$ the \textit{weak mixing} angle or \textit{Weinberg} angle. This angle is a single fundamental parameter and is defined as:
\begin{equation}
  g_W = \frac{g_e}{sin(\theta_W)}, \quad g_Z = \frac{g_e}{sin(\theta_W) cos(\theta_W)}
\end{equation}
with $g_W$, $g_Z$ and $g_e$ the weak and electromagnetic coupling constant. The masses of the $W^{+/-}$ and $Z^0$ bosons are related as $M_W = M_Z \cdot cos(\theta_W)$.

The last particle to describe is the Higgs Boson ($H$) of spin 0. This boson is the cornerstone of the Standard Model. The principle of gauge invariance requires that bosons are massless which is fulfilled for the photon and gluon but the $W^{+/-}$ and $Z^0$ bosons are massive particles. This means that $SU(2)_{L} \times U(1)_{Y}$ must be broken to generate masses for the heavy bosons. This is called the \textit{Electroweak Symmetry Breaking}. By introducing the Higgs boson, this breaks the $SU(2)_{L} \times U(1)_{Y}$ group by the \textit{Higgs mechanism} and generate masses for fermions and bosons. The theory \cite{Higgs:1964pj, Englert:1964et} has been formalized by P. Higgs, F. Englert and R. Brout$^\dagger$\footnote{Diseased in 2011} in 1964 and the particle has been discovered in 2012 at the LHC \cite{Aad:2012tfa, Chatrchyan:2012xdj}. They were awarded the Nobel Physics Prize in 2013 for their theory.

\section{The Higgs Boson}
\label{sec:HiggsTheo}

In the Standard Model, the Lagrangian $\mathcal{L}_{D}$, also known as the \textit{Dirac Lagrangian}, for free particle of spin 1/2 such as the electron is
\begin{equation}
  \mathcal{L}_{D} = i(\hbar c)\overline{\rm \Phi}\gamma^{\mu}\partial_{\mu}\Phi - (mc^2)\overline{\rm \Phi}\Phi
\end{equation}
with $\Phi$ a spinor field, $\gamma^{\mu}$ the Dirac $\gamma$-matrices (4 matrices) \cite{Peskin:1995ev} and $m$ the mass of the particle. This Lagrangian is invariant under the \textit{global} gauge transformation $\Phi \rightarrow e^{i\theta}\Phi$ with $\theta$ being constant. However, this in not invariant under \textit{local} gauge transformation where the phase factor $\theta$ is function of space-time $\theta(x)$.

The local gauge invariance can be restored by adding a new field $A_{\mu}$ which follows the local gauge transformation $A_{\mu} \rightarrow A_{\mu} + \partial_{\mu}\lambda$ and changing the derivative $\partial_{\mu}$ by the \textit{covariant derivative} $\mathcal{D}_{\mu} = \partial_{\mu} + i\frac{q}{\hbar c}A_{\mu}$. The local gauge invariant Lagrangian for a free particle of spin 1/2 is then:
\begin{equation}
  \mathcal{L}_{QED} = i(\hbar c)\overline{\rm \Phi}\gamma^{\mu}\mathcal{D}_{\mu}\Phi - (mc^2)\overline{\rm \Phi}\Phi - q\overline{\rm \Phi}\gamma^{\mu}\Phi A_{\mu} - \frac{1}{16\pi}F^{\mu\nu}F_{\mu\nu}
\end{equation}
with $F^{\mu\nu} \equiv \partial^{\mu}A^{\nu} - \partial^{\nu}A^{\mu}$ the photon field. The term $q\overline{\rm \Phi}\gamma^{\mu}\Phi A_{\mu}$ describes the interaction of the fermion with the photon. By introducing the new field $A_{\mu}$, a term $mA^{\mu}A_{\mu}$ should be present but breaks the local gauge symmetry thus the new field is massless (the photon). This Lagrangian describes well the fermions fields interacting with the photon field otherwise known as QED.

A similar approach was done by \textit{Yang-Mills} for the weak interaction, also for the strong interaction in QCD under $SU(3)$. It requires the Lagrangian to be invariant under $SU(2)_{L}$ local gauge transformation. This is done by changing the derivative with the according \textit{covariant derivative} to satisfy local gauge invariance under $SU(2)_{L}$.

\subsection{The Higgs mechanism}

\subsection{Interpretation in the SM}

\section{Quantum Chromodynamics}
\label{sec:QCD}

\subsection{Quarks and Hadronisation}

\section{Beyond the Standard Model}
\label{sec:BeyondSM}

\subsection{Short-comings of the SM}

\subsection{Supersymmetry (SUSY)}
